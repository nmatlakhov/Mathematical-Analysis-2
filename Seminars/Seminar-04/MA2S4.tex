\documentclass[12pt]{article}
\usepackage[left=1cm, right=1cm, top=2cm,bottom=1.5cm]{geometry} 

\usepackage[parfill]{parskip}
\usepackage[utf8]{inputenc}
\usepackage[T2A]{fontenc}
\usepackage[russian]{babel}
\usepackage{enumitem}
\usepackage[normalem]{ulem}
\usepackage{amsfonts, amsmath, amsthm, amssymb, mathtools,xcolor}
\usepackage{blkarray}

\usepackage{tabularx}
\usepackage{hhline}

\usepackage{accents}
\usepackage{fancyhdr}
\pagestyle{fancy}
\renewcommand{\headrulewidth}{1.5pt}
\renewcommand{\footrulewidth}{1pt}

\usepackage{graphicx}
\usepackage[figurename=Рис.]{caption}
\usepackage{subcaption}
\usepackage{float}

%%Наименование папки откуда забирать изображения
\graphicspath{ {./images/} }

%%Изменение формата для ввода доказательства
\renewcommand{\proofname}{$\square$  \nopunct}
\renewcommand\qedsymbol{$\blacksquare$}

%%Изменение отступа на таблицах
\addto\captionsrussian{%
	\renewcommand{\proofname}{$\square$ \nopunct}%
}
%% Римские цифры
\newcommand{\RN}[1]{%
	\textup{\uppercase\expandafter{\romannumeral#1}}%
}

%% Для удобства записи
\newcommand{\MR}{\mathbb{R}}
\newcommand{\MC}{\mathbb{C}}
\newcommand{\MQ}{\mathbb{Q}}
\newcommand{\MN}{\mathbb{N}}
\newcommand{\MZ}{\mathbb{Z}}
\newcommand{\MTB}{\mathbb{T}}
\newcommand{\MTI}{\mathbb{I}}
\newcommand{\MI}{\mathrm{I}}
\newcommand{\MCI}{\mathcal{I}}
\newcommand{\MJ}{\mathrm{J}}
\newcommand{\MH}{\mathrm{H}}
\newcommand{\MT}{\mathrm{T}}
\newcommand{\MU}{\mathcal{U}}
\newcommand{\MV}{\mathcal{V}}
\newcommand{\MB}{\mathcal{B}}
\newcommand{\MF}{\mathcal{F}}
\newcommand{\MW}{\mathcal{W}}
\newcommand{\ML}{\mathcal{L}}
\newcommand{\MP}{\mathcal{P}}
\newcommand{\VN}{\varnothing}
\newcommand{\VE}{\varepsilon}
\newcommand{\dx}{\, dx}
\newcommand{\dy}{\, dy}
\newcommand{\dz}{\, dz}
\newcommand{\dd}{\, d}


\theoremstyle{definition}
\newtheorem{defn}{Опр:}
\newtheorem{rem}{Rm:}
\newtheorem{prop}{Утв.}
\newtheorem{exrc}{Упр.}
\newtheorem{problem}{Задача}
\newtheorem{lemma}{Лемма}
\newtheorem{theorem}{Теорема}
\newtheorem{corollary}{Следствие}

\newenvironment{cusdefn}[1]
{\renewcommand\thedefn{#1}\defn}
{\enddefn}

\DeclareRobustCommand{\divby}{%
	\mathrel{\text{\vbox{\baselineskip.65ex\lineskiplimit0pt\hbox{.}\hbox{.}\hbox{.}}}}%
}
%Короткий минус
\DeclareMathSymbol{\SMN}{\mathbin}{AMSa}{"39}
%Длинная шапка
\newcommand{\overbar}[1]{\mkern 1.5mu\overline{\mkern-1.5mu#1\mkern-1.5mu}\mkern 1.5mu}
%Функция знака
\DeclareMathOperator{\sgn}{sgn}

%Функция ранга
\DeclareMathOperator{\rk}{\text{rk}}
\DeclareMathOperator{\diam}{\text{diam}}


%Обозначение константы
\DeclareMathOperator{\const}{\text{const}}

\DeclareMathOperator{\codim}{\text{codim}}

\DeclareMathOperator*{\dsum}{\displaystyle\sum}
\newcommand{\ddsum}[2]{\displaystyle\sum\limits_{#1}^{#2}}

%Интеграл в большом формате
\DeclareMathOperator{\dint}{\displaystyle\int}
\newcommand{\ddint}[2]{\displaystyle\int\limits_{#1}^{#2}}
\newcommand{\ssum}[1]{\displaystyle \sum\limits_{n=1}^{\infty}{#1}_n}

\newcommand{\smallerrel}[1]{\mathrel{\mathpalette\smallerrelaux{#1}}}
\newcommand{\smallerrelaux}[2]{\raisebox{.1ex}{\scalebox{.75}{$#1#2$}}}

\newcommand{\smallin}{\smallerrel{\in}}
\newcommand{\smallnotin}{\smallerrel{\notin}}

\newcommand*{\medcap}{\mathbin{\scalebox{1.25}{\ensuremath{\cap}}}}%
\newcommand*{\medcup}{\mathbin{\scalebox{1.25}{\ensuremath{\cup}}}}%

\makeatletter
\newcommand{\vast}{\bBigg@{3.5}}
\newcommand{\Vast}{\bBigg@{5}}
\makeatother

%Промежуточное значение для sup\inf, поскольку они имеют разную высоту
\newcommand{\newsup}{\mathop{\smash{\mathrm{sup}}}}
\newcommand{\newinf}{\mathop{\mathrm{inf}\vphantom{\mathrm{sup}}}}

%Скалярное произведение
\newcommand{\inner}[2]{\left\langle #1, #2 \right\rangle }
\newcommand{\linsp}[1]{\left\langle #1 \right\rangle }
\newcommand{\linmer}[2]{\left\langle #1 \vert #2\right\rangle }

%Подпись символов снизу
\newcommand{\ubar}[1]{\underaccent{\bar}{#1}}

%% Шапка для букв сверху
\newcommand{\wte}[1]{\widetilde{#1}}
\newcommand{\wht}[1]{\widehat{#1}}

%%Трансформация Фурье
\newcommand{\fourt}[1]{\mathcal{F}\left(#1\right)}
\newcommand{\ifourt}[1]{\mathcal{F}^{-1}\left(#1\right)}

%%Символ вектора
\newcommand{\vecm}[1]{\overrightarrow{#1\,}}

%%Пространстов матриц
\newcommand{\mat}[2]{\operatorname{Mat}_{#1\times #2}}


%%Взятие в скобки, модули и норму
\newcommand{\parfit}[1]{\left( #1 \right)}
\newcommand{\modfit}[1]{\left| #1 \right|}
\newcommand{\sqparfit}[1]{\left\{ #1 \right\}}
\newcommand{\normfit}[1]{\left\| #1 \right\|}

%%Функция для обозначения равномерной сходимости по множеству
\newcommand{\uconv}[1]{\overset{#1}{\rightrightarrows}}
\newcommand{\uconvm}[2]{\overset{#1}{\underset{#2}{\rightrightarrows}}}


%%Функция для обозначения нижнего и верхнего интегралов
\def\upint{\mathchoice%
	{\mkern13mu\overline{\vphantom{\intop}\mkern7mu}\mkern-20mu}%
	{\mkern7mu\overline{\vphantom{\intop}\mkern7mu}\mkern-14mu}%
	{\mkern7mu\overline{\vphantom{\intop}\mkern7mu}\mkern-14mu}%
	{\mkern7mu\overline{\vphantom{\intop}\mkern7mu}\mkern-14mu}%
	\int}
\def\lowint{\mkern3mu\underline{\vphantom{\intop}\mkern7mu}\mkern-10mu\int}

%%След матрицы
\DeclareMathOperator*{\tr}{tr}

\makeatletter
\renewcommand*\env@matrix[1][*\c@MaxMatrixCols c]{%
	\hskip -\arraycolsep
	\let\@ifnextchar\new@ifnextchar
	\array{#1}}
\makeatother


%% Переопределение функции хи, чтобы выглядела более приятно
\makeatletter
\@ifdefinable\@latex@chi{\let\@latex@chi\chi}
\renewcommand*\chi{{\@latex@chi\smash[t]{\mathstrut}}} % want only bottom half of \mathstrut
\makeatletter

\begin{document}
\lhead{Математический анализ - \RN{2}}
\chead{Косухин О.Н.}
\rhead{Семинар - 4}
\section*{Интегрирование рациональных функций}

В прошлый раз мы рассмотрели многочлены вида:
$$
	\deg{P} < n, \, \dfrac{P(x)}{(x - z_1)(x - z_2)\dotsc(x - z_n)} = \dfrac{P(z_1)}{Q'(z_1)}{\cdot}\dfrac{1}{x - z_1} + \dotsc + \dfrac{P(z_n)}{Q'(z_n)}{\cdot}\dfrac{1}{x - z_n}
$$
где $Q(x) = (x - z_1)(x - z_2)\dotsc(x - z_n)$ и все корни различны. Также, возник вопрос, умеем ли мы интегрировать функции вида:
$$
	\dfrac{1}{x - z_j}, \, z_j \in \MC
$$
Это можно сделать. Попробуем понять принцип:
$$
	\dint \dfrac{1}{x}dx = \ln{|x|} + C, \, \dint \dfrac{1}{x - a}dx = \ln{|x - a|} + C, \, \dint\dfrac{1}{x - ai}dx = ? = \ln{|x - ai|} + C
$$
Что такое модуль: $|x - ai| = \sqrt{x^2 + a^2}$. Рассмотрим логарифм этого модуля:
$$
	\left(\ln{\left(\sqrt{x^2 + a^2}\right)}\right)' = \left(\dfrac{1}{2}\ln{\left(x^2 + a^2\right)}\right)' = \dfrac{1}{2}{\cdot}\dfrac{2x}{x^2 + a^2} = \dfrac{x}{x^2 + a^2} \neq \dfrac{1}{x - ai} = \dfrac{x + ai}{x^2 + a^2}
$$
$$
	\dfrac{ai}{x^2 + a^2} = i{\cdot}\left(\arctg{\left(\dfrac{x}{a}\right)}\right)' = i{\cdot}\dfrac{1}{a}{\cdot}\dfrac{1}{1 + \frac{x^2}{a^2}} = \dfrac{ai}{x^2 + a^2}
$$
$$
	\dint \dfrac{1}{x - ai}dx = \ln{|x - ai|} + i{\cdot}\arctg{\left(\dfrac{x}{a}\right)} + C
$$
\begin{rem}
	Логарифм для комплексных переменных это многозначная функция, она принимает сразу бесконечно много разных значений. Поэтому мы будем ими пользоваться, только если нам это будет выгодно.
\end{rem}

\begin{problem}(\textbf{Д1884})
	$$
		\dint \dfrac{dx}{x^4 + 1}
	$$
\end{problem}
\begin{proof}
	У этого многочлена нет действительных корней. Попробуем разложить этот многочлен используя формулу суммы квадратов:
	$$
		a^2 + b^2 = a^2 + b^2 +2ab - 2ab =(a + b)^2 - 2ab = (a + b + \sqrt{2ab})(a + b - \sqrt{2ab}), \, ab > 0
	$$
	$$
		x^4 + 1 = (x^2 + 1 + \sqrt{2x^2})(x^2 + 1 - \sqrt{2x^2})  = 
	$$
	$$
		= (x^2 + 1 + \sqrt{2}|x|)(x^2 + 1 - \sqrt{2}|x|) = (x^2 + 1 + x\sqrt{2})(x^2 + 1 - x\sqrt{2})
	$$
	Поскольку при раскрытии модуля, слагаемые просто поменяются местами в случае $x <0$. Попробуем проинтегрировать методом неопределенных коэффициентов:
	$$
		\dfrac{1}{(x^2 + \sqrt{2}x + 1)(x^2 - \sqrt{2}x + 1)} = \dfrac{Ax + B}{x^2 + \sqrt{2}x + 1} + \dfrac{Cx + D}{x^2 - \sqrt{2}x + 1}
	$$
	$$
		(Ax + B)(x^2 - \sqrt{2}x + 1) + (Cx + D)(x^2 + \sqrt{2}x + 1) = 1
	$$
	Рассмотрим коэффициенты по степеням:
	$$
		\left\{
			\begin{matrix}
				x^3 \colon& &&&&A &+& C &=& 0\\
				x^2 \colon& B &+& D &-& \sqrt{2}A &+& \sqrt{2}C &=& 0\\
				x \colon &A &-& B\sqrt{2} &+& C &+& D\sqrt{2} &=& 0\\
				1 \colon &  &&&&B &+& D &=& 1
			\end{matrix}
		\right. \Rightarrow
		\left\{
			\begin{matrix}
				A &+& C &=& 0 \\
				\sqrt{2}C &-& \sqrt{2}A &=& -1\\
				&&D &=& B\\
				B &+& D &=& 1 
			\end{matrix}
		\right. \Rightarrow
		\left\{
		\begin{matrix}
			A &=& \tfrac{1}{2\sqrt{2}}\\[5pt]
			C &=& -\tfrac{1}{2\sqrt{2}}\\[5pt]
			B &=& \tfrac{1}{2}\\[5pt]
			D &=& \tfrac{1}{2}
		\end{matrix}
		\right.
	$$
	$$
		\dint \dfrac{\frac{1}{2\sqrt{2}}x + \frac{1}{2}}{x^2 + \sqrt{2}x + 1}dx + \dint \dfrac{-\frac{1}{2\sqrt{2}}x + \frac{1}{2}}{ x^2 - \sqrt{2}x + 1}dx
	$$
	Как считать такие интегралы? Очень легко, если пользоваться следующей схемой:
	$$
		\dint \dfrac{2ax + b}{ax^2 + bx + c}dx = \dint \dfrac{f'(x)}{f(x)}dx = \ln{|f(x)|} + C
	$$
	Отношение под интегралом называется \uwave{логарифмической производной}. Рассмотрим наши функции:
	$$
		(x^2 \pm \sqrt{x} + 1)' = 2x \pm \sqrt{2} \Rightarrow \dfrac{1}{2\sqrt{2}}x + \dfrac{1}{2} = \dfrac{1}{4\sqrt{2}}(2x + \sqrt{2}) + \dfrac{1}{4}
	$$
	Поскольку наше соотношение не является логарифмической производной, то мы хотим их разложить на отношения вида:
	$$
		\dfrac{af'(x) + b}{f(x)} \Rightarrow \dint \dfrac{\frac{1}{2\sqrt{2}}x + \frac{1}{2}}{x^2 + \sqrt{2}x + 1}dx = \dfrac{1}{4\sqrt{2}}\dint \dfrac{2x + \sqrt{2}}{x^2 + \sqrt{2}x + 1}dx + \dfrac{1}{4}\dint\dfrac{1}{x^2 + \sqrt{2}x + 1}dx =
	$$
	$$
		= \dfrac{1}{4\sqrt{2}}	\ln{(x^2 +\sqrt{2}x + 1)} + \dfrac{1}{4}\dint \dfrac{1}{\left(x + \frac{\sqrt{2}}{2}\right)^2+ \frac{1}{2}}dx =  \dfrac{1}{4\sqrt{2}}\ln{(x^2 +\sqrt{2}x + 1)} + \dfrac{\sqrt{2}}{4}\arctg{\left(\sqrt{2}x + 1\right)} + C
	$$
	где в логарифме не нужен модуль, поскольку квадратный трехчлен всегда положителен. Со вторым интегралом всё аналогично:
	$$
		T(x) = \dfrac{\frac{1}{2\sqrt{2}}x + \frac{1}{2}}{x^2 + \sqrt{2}x + 1} \Rightarrow T(-x) = \dfrac{-\frac{1}{2\sqrt{2}}x + \frac{1}{2}}{x^2 - \sqrt{2}x + 1}, \, F'(x) = T(x) \Rightarrow T(-x) = -F'(-x)
	$$
	Тогда мы просто перепишем ответ:
	$$
		\dint \dfrac{-\frac{1}{2\sqrt{2}}x + \frac{1}{2}}{x^2 - \sqrt{2}x + 1} dx = - \dfrac{1}{4\sqrt{2}}\ln{(x^2 -\sqrt{2}x + 1)} - \dfrac{\sqrt{2}}{4}\arctg{\left(-\sqrt{2}x + 1\right)} + C
	$$
\end{proof}

\begin{exrc}(\textbf{Задача из теории чисел}) 
	Доказать, что число $4^p + p^4$ - составное, где $p$ - простое число.
\end{exrc}
\begin{proof}
	Необходимо использовать формулу суммы квадратов.
\end{proof}

\newpage
Что будет, если степень числителя больше степени знаменателя?
\begin{problem}(\textbf{Д1868})
	$$
		\dint \dfrac{x^{10} dx}{x^2 + x -2}
	$$
\end{problem}
\begin{proof}
	Поскольку $\deg{x^{10}} > \deg{x^2 + x - 2}$, то сначала надо поделить многочлен в числителе на многочлен в знаменателе с остатком. Можно это сделать двумя способами, первый из них - в столбик:
	$$
		{\cdot}x^8 \colon x^{10} - x^{10} - x^9 + 2x^8 = -x^9 + 2x^8
	$$
	$$
		{\cdot}(-x^7) \colon -x^9 + 2x^8 + x^9 + x^8 - 2x^7 = 3x^8 - 2x^7
	$$
	И так далее. Попробуем сделать это по-другому. Трехчлен в знаменателе легко раскладывается на множители:
	$$
		x^2 + x - 2 = (x - 1)(x + 2)
	$$
	На линейные множители делить гораздо проще. Воспользуемся схемой Горнера (первая ячейка умножается на ячейку степени и складываются с пониженной степенью выше):
	$$
		\begin{matrix}[c|c|c|c|c|c|c|c|c|c|c|c|]
			 & 10 & 9 & 8 & 7 & 6 & 5 & 4 & 3 & 2 & 1 & 0\\\hline
			 & 1  & 0 & 0 & 0 & 0 & 0 & 0 & 0 & 0 & 0 & 0 \\ \hline
			1& 1  & 1 & 1 & 1 & 1 & 1 & 1 & 1 & 1 & 1 & 1  \\ \hline
		\end{matrix}
	$$
	Таким образом, мы получаем:
	$$
		(x^9 + x^8 + x^7 + x^6 + x^5 + x^4 + x^3 + x^2 + x + 1)(x-1) + 1 = x^{10}
	$$
	Это по сути частный случай формулы сокращенного умножения:
	$$
		x^n - 1 = (x- 1)(x^{n-1} + x^{n-2} + \dotsc + x + 1 )
	$$
	Поделим полученный многочлен на $(x + 2)$:
	$$
		\begin{matrix}[c|c|c|c|c|c|c|c|c|c|c|c|]
			& 9 & 8 & 7 & 6 & 5 & 4 & 3 & 2 & 1 & 0\\\hline
			& 1  & 1 & 1 & 1 & 1 & 1 & 1 & 1 & 1 & 1 \\ \hline
			-2& 1  & -1 & 3 & -6 + 1 = -5 & 10 + 1 & -22 + 1 & 43 & -85 & 171 & -341  \\ \hline
		\end{matrix}
	$$
	Тогда:
	$$
		x^{10} = 1 + (x- 1)((x+2)(x^8 - x^7 + 3x^6 - 5x^5 + 11x^4 - 21 x^3 + 43x^2 -85 x + 171 ) - 341)
	$$
	Поделим полученный многочлен в нашей дроби:
	$$
		\dfrac{x^{10}}{x^2 + x -2 } = \dfrac{1}{(x - 1)(x+2)} + x^8 - x^7 + 3x^6 - 5x^5 + 11x^4 - 21 x^3 + 43x^2 -85 x + 171 - \dfrac{341}{x+2}
	$$
	Отсюда, каждое слагаемое в отдельности легко интегрируется:
	$$
		\dfrac{1}{(x- 1)(x+2)} = \dfrac{1}{3}\dfrac{1}{x - 1} - \dfrac{1}{3}\dfrac{1}{x + 2} \Rightarrow  \dint \dfrac{x^{10} dx}{x^2 + x -2} = \dfrac{1}{3}\ln{|x-1|} - \dfrac{1}{3}\ln{|x +2|} + 
	$$
	$$
		+ \dfrac{1}{9}x^9 - \dfrac{1}{8}x^8 + \dfrac{3}{7}x^7 - \dfrac{5}{6}x^6 + \dfrac{11}{5}x^5 - \dfrac{21}{4}x^4 + \dfrac{43}{3}x^3 - \dfrac{85}{2}x^2 + 171x - 341\ln{|x + 2|} + C
	$$
\end{proof}
\newpage
\section*{Метод Остроградского}
Он применяется тогда, когда у знаменателя рациональной дроби есть кратные корни. Пусть мы хотим посчитать интеграл:
$$
	\dint \dfrac{P(x)}{Q(x)}dx, \, Q(x) = \prod\limits_{j}(x - a_j)^{k_j}{\cdot}\prod\limits_{m}(x^2 + p_m x + q_m)^{l_m}
$$
Пусть у $Q(x)$ есть рациональные корни, построим вспомогательные многочлены:
$$
	Q_1(x) = \prod\limits_{j}(x - a_j){\cdot}\prod\limits_{m}(x^2 + p_m x + q_m), \, Q_2(x) = \prod\limits_{j}(x - a_j)^{k_j-1}{\cdot}\prod\limits_{m}(x^2 + p_m x + q_m)^{l_m-1}
$$
$$
	Q(x) = Q_1(x){\cdot}Q_2(x)
$$
Тогда найдутся многочлены $P_1(x)$ и $P_2(x)$ такие, что:
$$
	\dint \dfrac{P(x)}{Q(x)}dx = \dfrac{P_2(x)}{Q_2(x)} + \dint \dfrac{P_1(x)}{Q_1(x)}dx, \, \deg{P_2} < \deg{Q_2}, \, \deg{P_1} < \deg{Q_1}
$$
\begin{theorem}(\textbf{метод Остроградского})
	Существуют $P_1(x), P_2(x)$ такие, что верно равенство:
	$$
		\dint \dfrac{P(x)}{Q(x)}dx = \dfrac{P_2(x)}{Q_2(x)} + \dint \dfrac{P_1(x)}{Q_1(x)}dx, \, \deg{P_2} < \deg{Q_2}, \, \deg{P_1} < \deg{Q_1}
	$$
\end{theorem}

Возьмем производные от обеих частей:
$$
	\dfrac{P(x)}{Q(x)} = \dfrac{P'_2(x)Q_2(x) - P_2(x)Q'_2(x)}{Q_2^2(x)} + \dfrac{P_1(x)}{Q_1(x)} 
$$
Умножим левую и правую части на $Q_1{\cdot}Q_2^2$, тогда мы получим:
$$
	\dfrac{P(x){\cdot}Q_1(x){\cdot}Q_2^2(x)}{Q_1(x){\cdot}Q_2(x)} = Q_1(x){\cdot}Q_2^2(x){\cdot}\left( \dfrac{P'_2(x)Q_2(x) - P_2(x)Q'_2(x)}{Q_2^2(x)} + \dfrac{P_1(x)}{Q_1(x)} \right) \Rightarrow
$$
$$
	\Rightarrow P(x){\cdot}Q_2(x) = P'_2(x){\cdot}Q(x) - P_2(x){\cdot}Q'_2(x){\cdot}Q_1(x) + P_1(x){\cdot}Q_2^2(x)
$$
Оказывается, что из этого равенства всегда можно восстановить многочлены $P_1(x)$ и $P_2(x)$.


\begin{problem}(\textbf{Д1891})
	$$
		\dint \dfrac{xdx}{(x-1)^2(x + 1)^3}
	$$
\end{problem}
\begin{proof}
	$$
		Q(x) = (x-1)^2(x+1)^3, \, Q_1(x) = (x-1)(x + 1), \, Q_2(x) = (x-1)(x+1)^2
	$$
	$$
		Q'_2(x) = (x+1)^2 + 2(x+1)(x - 1) = (x + 1)(3x -1 )
	$$
	$$
		\dint \dfrac{xdx}{(x-1)^2(x + 1)^3} = \dfrac{Ax^2 + Bx + C}{(x-1)(x+1)^2} + \dint \dfrac{Dx + E}{(x-1)(x+1)}dx
	$$
	Запишем формулу выше для нашего случая:
	$$
		x(x-1)(x+1)^2 = (2Ax + B)(x-1)^2(x+1)^3 - (Ax^2 + Bx + C)(x+1)^2(3x-1)(x-1) +
	$$
	$$
		+ (Dx + E)(x-1)^2(x+1)^4 \Rightarrow /(x-1)(x+1)^2 \Rightarrow
	$$
	$$	
		\Rightarrow x = (2Ax + B)(x-1)(x+1) - (Ax^2 + Bx + C)(3x -1) + (Dx + E)(x-1)(x+1)^2
	$$
	Иногда удобнее находить коэффициенты, подставив в левую/правую части $1$:
	$$
		x = 1 \Rightarrow 1 = -(A +B +C)2, \, x = -1 \Rightarrow -1 = -(A - B + C)(-4)
	$$
	$$
		x = 0 \Rightarrow 0 = -B +C -E
	$$
	Не хватает ещё двух уравнений $\Rightarrow$ возьмем их из старших степеней:
	$$
		x^4 \colon 0= D, \, x^3 \colon 0 = 2A -3A +E
	$$
	Таким образом, мы получаем систему:
	$$
		\left\{
			\begin{matrix}
				A &+& B &+& C &=& -\tfrac{1}{2}\\[5pt]
				A &-& B &+& C &=& -\tfrac{1}{4}\\[5pt]
				-B &+& C &-& E &=& 0\\[5pt]
				&&&&E &=& A
			\end{matrix}
		\right. \Rightarrow
		\left\{
			\begin{matrix}
				&&B & = & -\tfrac{1}{8}\\[5pt]
				A &+& C &=& -\tfrac{3}{8}\\[5pt]
				C &-& A &=& -\tfrac{1}{8}\\[5pt]
				&&A &=& E
			\end{matrix}
		\right. \Rightarrow
		\left\{
			\begin{matrix}
				A & = & -\tfrac{1}{8}\\[5pt]
				B &=& -\tfrac{1}{8}\\[5pt]
				C &=& -\tfrac{1}{4}\\[5pt]
				E &=& -\tfrac{1}{8}
			\end{matrix}
		\right.
	$$
	$$
		\dint \dfrac{xdx}{(x-1)^2(x + 1)^3} = \dfrac{-\frac{1}{8}x^2  - \frac{1}{8}x - \frac{1}{4}}{(x-1)(x+1)^2} + \dint \dfrac{-\frac{1}{8}}{(x-1)(x+1)}dx
	$$
	$$
		\dfrac{-\frac{1}{8}}{(x-1)(x+1)} = -\dfrac{1}{8}{\cdot}\left(\dfrac{1}{2}\left(\dfrac{1}{x-1} - \dfrac{1}{x+1}\right)\right) \Rightarrow
	$$
	$$
		\Rightarrow \dint \dfrac{xdx}{(x-1)^2(x + 1)^3} = -\dfrac{1}{8}\dfrac{x^2  + x + 2}{(x-1)(x+1)^2} - \dfrac{1}{16} \ln{\left|\dfrac{x-1}{x + 1}\right|} +C
	$$
\end{proof}

Отметим, что в методе Остроградского:
$$
	\dint \dfrac{P(x)}{Q(x)}dx = \dfrac{P_2(x)}{Q_2(x)} + \dint \dfrac{P_1(x)}{Q_1(x)}dx, \, \deg{P_2} < \deg{Q_2}, \, \deg{P_1} < \deg{Q_1}
$$
Часть $\frac{P_2(x)}{Q_2(x)}$ называется \uwave{алгебраической}, а часть $\int \frac{P_1(x)}{Q_1(x)}dx$ называется \uwave{трансцендентной}. Интегрирование трансцендентной части обычно дает сумму логарифмов и арктангенсов: логарифмы - если есть линейные множители в знаменателе $Q_1$, арктангенсы - если есть квадратичные множители.
\begin{rem}
	Иногда можно найти алгебраическую часть, не разбирая трансцендентную, поскольку на бесконечности трансцендентная часть, состоящая из логарифмов и арктангенсов это $o$-малое от алгебраической части.
\end{rem}

\begin{problem}(\textbf{Д1892})
	$$
		\dint \dfrac{dx}{(x^3 + 1)^2 }
	$$
\end{problem}
\begin{proof}
	$$
		Q(x) = (x^3 + 1)^2
	$$
	Заметим, что у $x^3 + 1$ нет кратных корней, это обычно можно понять так:
	$$
		G(x) = (x- x_1)^k\dotsc \Rightarrow G'(x) = n(x - x_1)^{k-1}\dotsc
	$$
	то есть надо посмотреть, есть ли у многочлена и его производной общие корни, поскольку кратные корни понижаются в степени, но остаются. В нашем случае:
	$$
		G(x) = x^3 + 1 = 0, \, G'(x) = 3x^2 = 0
	$$
	То есть система не имеет решений $\Rightarrow$ кратных корней нет. Тогда:
	$$
		Q_1(x) = x^3 + 1, \, Q_2(x) = x^3 + 1
	$$
	$$
		\dint \dfrac{dx}{(x^3 + 1)^2 } = \dfrac{Ax^2 + Bx + C}{x^3 +1} + \dint\dfrac{Dx^2 + Ex + F}{x^3 + 1}dx
	$$
	$$
		1(x^3 + 1) = (2Ax + B)(x^3 + 1)^2 - (Ax^2 + Bx + C)3x^2(x^3 + 1) + (Dx^2 + Ex + F)(x^3 +1)^2 \Rightarrow
	$$
	$$
		\Rightarrow 1 = (2Ax + B) (x^3+1) - (Ax^2 + Bx +C)3x^2 +(Dx^2 + Ex +F)(x^3 +1)
	$$
	Как можно быстро получить коэффициенты в левой и правой частях? Рассмотрим ещё один метод:
	$$
		z \in \MC \colon z^3 + 1 = 0 \Rightarrow z_1 = -1, \, z_2 = e^{\frac{\pi i}{3}}, \, z_3 = e^{-\frac{\pi i}{3}} \Rightarrow
	$$
	$$
		\Rightarrow 1 = -(Az^2 + Bz + C)3z^2 = 3Az + 3B - 3Cz^2 = 0
	$$
	Многочлен степени не выше $2$ в трёх точках равен $1 \Rightarrow$ этот многочлен $\equiv 1$, тогда:
	$$
		A = 0, \, C = 0, \, B = \dfrac{1}{3} 
	$$
	Таким образом, мы посчитали алгебраическую часть не считая трансцендентную. Можем досчитать остальные коэффициенты и найти исходный интеграл.
\end{proof}

\textbf{ДЗ}: решить методом Остроградского: $1873, 1892, 1894, 1898$ (только нужно алгебраическую часть); без метода Остроградского: $1883$.

\end{document}