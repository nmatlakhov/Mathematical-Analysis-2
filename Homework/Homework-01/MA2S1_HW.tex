\documentclass[12pt]{article}
\usepackage[left=1cm, right=1cm, top=2cm,bottom=1.5cm]{geometry} 

\usepackage[parfill]{parskip}
\usepackage[utf8]{inputenc}
\usepackage[T2A]{fontenc}
\usepackage[russian]{babel}
\usepackage{enumitem}
\usepackage[normalem]{ulem}
\usepackage{amsfonts, amsmath, amsthm, amssymb, mathtools,xcolor}
\usepackage{blkarray}

\usepackage{tabularx}
\usepackage{hhline}

\usepackage{accents}
\usepackage{fancyhdr}
\pagestyle{fancy}
\renewcommand{\headrulewidth}{1.5pt}
\renewcommand{\footrulewidth}{1pt}

\usepackage{graphicx}
\usepackage[figurename=Рис.]{caption}
\usepackage{subcaption}
\usepackage{float}

%%Наименование папки откуда забирать изображения
\graphicspath{ {./images/} }

%%Изменение формата для ввода доказательства
\renewcommand{\proofname}{$\square$  \nopunct}
\renewcommand\qedsymbol{$\blacksquare$}

%%Изменение отступа на таблицах
\addto\captionsrussian{%
	\renewcommand{\proofname}{$\square$ \nopunct}%
}
%% Римские цифры
\newcommand{\RN}[1]{%
	\textup{\uppercase\expandafter{\romannumeral#1}}%
}

%% Для удобства записи
\newcommand{\MR}{\mathbb{R}}
\newcommand{\MC}{\mathbb{C}}
\newcommand{\MQ}{\mathbb{Q}}
\newcommand{\MN}{\mathbb{N}}
\newcommand{\MZ}{\mathbb{Z}}
\newcommand{\MTB}{\mathbb{T}}
\newcommand{\MTI}{\mathbb{I}}
\newcommand{\MI}{\mathrm{I}}
\newcommand{\MCI}{\mathcal{I}}
\newcommand{\MJ}{\mathrm{J}}
\newcommand{\MH}{\mathrm{H}}
\newcommand{\MT}{\mathrm{T}}
\newcommand{\MU}{\mathcal{U}}
\newcommand{\MV}{\mathcal{V}}
\newcommand{\MB}{\mathcal{B}}
\newcommand{\MF}{\mathcal{F}}
\newcommand{\MW}{\mathcal{W}}
\newcommand{\ML}{\mathcal{L}}
\newcommand{\MP}{\mathcal{P}}
\newcommand{\VN}{\varnothing}
\newcommand{\VE}{\varepsilon}

\theoremstyle{definition}
\newtheorem{defn}{Опр:}
\newtheorem{rem}{Rm:}
\newtheorem{prop}{Утв.}
\newtheorem{exrc}{Упр.}
\newtheorem{problem}{Задача}
\newtheorem{lemma}{Лемма}
\newtheorem{theorem}{Теорема}
\newtheorem{corollary}{Следствие}

\newenvironment{cusdefn}[1]
{\renewcommand\thedefn{#1}\defn}
{\enddefn}

\DeclareRobustCommand{\divby}{%
	\mathrel{\text{\vbox{\baselineskip.65ex\lineskiplimit0pt\hbox{.}\hbox{.}\hbox{.}}}}%
}
%Короткий минус
\DeclareMathSymbol{\SMN}{\mathbin}{AMSa}{"39}
%Длинная шапка
\newcommand{\overbar}[1]{\mkern 1.5mu\overline{\mkern-1.5mu#1\mkern-1.5mu}\mkern 1.5mu}
%Функция знака
\DeclareMathOperator{\sgn}{sgn}

%Функция ранга
\DeclareMathOperator{\rk}{\text{rk}}
\DeclareMathOperator{\diam}{\text{diam}}


%Обозначение константы
\DeclareMathOperator{\const}{\text{const}}

\DeclareMathOperator{\codim}{\text{codim}}

\DeclareMathOperator*{\dsum}{\displaystyle\sum}
\newcommand{\ddsum}[2]{\displaystyle\sum\limits_{#1}^{#2}}

%Интеграл в большом формате
\DeclareMathOperator{\dint}{\displaystyle\int}
\newcommand{\ddint}[2]{\displaystyle\int\limits_{#1}^{#2}}
\newcommand{\ssum}[1]{\displaystyle \sum\limits_{n=1}^{\infty}{#1}_n}

\newcommand{\smallerrel}[1]{\mathrel{\mathpalette\smallerrelaux{#1}}}
\newcommand{\smallerrelaux}[2]{\raisebox{.1ex}{\scalebox{.75}{$#1#2$}}}

\newcommand{\smallin}{\smallerrel{\in}}
\newcommand{\smallnotin}{\smallerrel{\notin}}

\newcommand*{\medcap}{\mathbin{\scalebox{1.25}{\ensuremath{\cap}}}}%
\newcommand*{\medcup}{\mathbin{\scalebox{1.25}{\ensuremath{\cup}}}}%

\makeatletter
\newcommand{\vast}{\bBigg@{3.5}}
\newcommand{\Vast}{\bBigg@{5}}
\makeatother

%Промежуточное значение для sup\inf, поскольку они имеют разную высоту
\newcommand{\newsup}{\mathop{\smash{\mathrm{sup}}}}
\newcommand{\newinf}{\mathop{\mathrm{inf}\vphantom{\mathrm{sup}}}}

%Скалярное произведение
\newcommand{\inner}[2]{\left\langle #1, #2 \right\rangle }
\newcommand{\linsp}[1]{\left\langle #1 \right\rangle }
\newcommand{\linmer}[2]{\left\langle #1 \vert #2\right\rangle }

%Подпись символов снизу
\newcommand{\ubar}[1]{\underaccent{\bar}{#1}}

%% Шапка для букв сверху
\newcommand{\wte}[1]{\widetilde{#1}}
\newcommand{\wht}[1]{\widehat{#1}}

%%Трансформация Фурье
\newcommand{\fourt}[1]{\mathcal{F}\left(#1\right)}
\newcommand{\ifourt}[1]{\mathcal{F}^{-1}\left(#1\right)}

%%Символ вектора
\newcommand{\vecm}[1]{\overrightarrow{#1\,}}

%%Пространстов матриц
\newcommand{\mat}[2]{\operatorname{Mat}_{#1\times #2}}


%%Взятие в скобки, модули и норму
\newcommand{\parfit}[1]{\left( #1 \right)}
\newcommand{\modfit}[1]{\left| #1 \right|}
\newcommand{\sqparfit}[1]{\left\{ #1 \right\}}
\newcommand{\normfit}[1]{\left\| #1 \right\|}

%%Функция для обозначения равномерной сходимости по множеству
\newcommand{\uconv}[1]{\overset{#1}{\rightrightarrows}}
\newcommand{\uconvm}[2]{\overset{#1}{\underset{#2}{\rightrightarrows}}}


%%Функция для обозначения нижнего и верхнего интегралов
\def\upint{\mathchoice%
	{\mkern13mu\overline{\vphantom{\intop}\mkern7mu}\mkern-20mu}%
	{\mkern7mu\overline{\vphantom{\intop}\mkern7mu}\mkern-14mu}%
	{\mkern7mu\overline{\vphantom{\intop}\mkern7mu}\mkern-14mu}%
	{\mkern7mu\overline{\vphantom{\intop}\mkern7mu}\mkern-14mu}%
	\int}
\def\lowint{\mkern3mu\underline{\vphantom{\intop}\mkern7mu}\mkern-10mu\int}

%%След матрицы
\DeclareMathOperator*{\tr}{tr}

\makeatletter
\renewcommand*\env@matrix[1][*\c@MaxMatrixCols c]{%
	\hskip -\arraycolsep
	\let\@ifnextchar\new@ifnextchar
	\array{#1}}
\makeatother


%% Переопределение функции хи, чтобы выглядела более приятно
\makeatletter
\@ifdefinable\@latex@chi{\let\@latex@chi\chi}
\renewcommand*\chi{{\@latex@chi\smash[t]{\mathstrut}}} % want only bottom half of \mathstrut
\makeatletter

\begin{document}
\lhead{Математический анализ - \RN{2}}
\chead{Косухин О.Н.}
\rhead{Семинар - 1: ДЗ}
\section*{Неопределенный интеграл}
\textbf{ДЗ}: $1635$, $1643$, $1646$ (вспомнить сумму кубов), $1650$, $1656$, $1659$, $1667$, $1672$.

\textbf{ДЗ}: найти $\th^{-1}(y)$, $\cth^{-1}(y)$ и посчитать их производные. 

\begin{problem}(\textbf{Д1635})
	$$
		\ddint{}{}\dfrac{(1 - x)^3}{x\sqrt[3]{x}}dx
	$$
\end{problem}
\begin{proof}
	$$
		\ddint{}{}\dfrac{(1 - x)^3}{x\sqrt[3]{x}}dx = \ddint{}{}\dfrac{1 - 3x + 3x^2 - x^3}{x\sqrt[3]{x}}dx = \ddint{}{}\dfrac{1}{x^{\frac{4}{3}}} - 3\dfrac{1}{x^{\frac{1}{3}}} + 3 x^{\frac{2}{3}} - x^{\frac{5}{3}}dx =
	$$
	$$	
		= -3 \dfrac{1}{\sqrt[3]{x}} - \dfrac{9}{2} x^{\frac{2}{3}} + \dfrac{9}{5} x^{\frac{5}{3}} - \dfrac{3}{10}x^{\frac{8}{3}}+ C = -\dfrac{3}{\sqrt[3]{x}}\left(1 + \dfrac{3}{2}x  -\dfrac{3}{5}x^2 + \dfrac{1}{8}x^3\right) + C
	$$
\end{proof}

\begin{problem}(\textbf{Д1643})
	$$
		\ddint{}{}\dfrac{\sqrt{x^2 +1} - \sqrt{x^2 - 1}}{\sqrt{x^4 - 1}}dx
	$$
\end{problem}
\begin{proof}
	Поскольку деление на корень из $x^4-1$, то промежуток подразумевается $(-\infty, -1) \cup (1, +\infty)$:
	$$
		\sqrt{x^4 - 1} = \sqrt{x^2  - 1}{\cdot}\sqrt{x^2 + 1} \Rightarrow 
	$$
	$$	
		\Rightarrow \ddint{}{}\dfrac{\sqrt{x^2 +1} - \sqrt{x^2 - 1}}{\sqrt{x^4 - 1}}dx = 	\ddint{}{}\dfrac{1}{\sqrt{x^2 - 1}}dx - \ddint{}{}\dfrac{1}{\sqrt{x^2 + 1}}dx =
	$$
	$$
		= \ln\left|x + \sqrt{ x^2-1}\right| - \ln\left|x + \sqrt{ x^2 + 1}\right| + C = \ln{\left|\dfrac{x + \sqrt{ x^2-1}}{x + \sqrt{ x^2 + 1}}\right|} + C
	$$
\end{proof}

\begin{problem}(\textbf{Д1646})
	$$
		\dint\dfrac{e^{3x} + 1}{e^x + 1}dx
	$$
\end{problem}
\begin{proof}
	$$
		\dint\dfrac{e^{3x} + 1}{e^x + 1}dx = \dint\dfrac{e^{x} + 1}{e^x + 1}{\cdot}(e^{2x} - e^x + 1)dx = \dfrac{1}{2}e^{2x} - e^x + x + C
	$$
\end{proof}

\begin{problem}(\textbf{Д1650})
	$$
		\dint \tg^2{x}dx
	$$
\end{problem}
\begin{proof}
	$$
		\dint \tg^2{x}dx = \dint (1 + \tg^2{x})dx - x + C = \tg{x} - x + C
	$$
\end{proof}
\newpage
\begin{problem}(\textbf{Д1656})
	$$
		\dint (2x + 3)^{10}dx
	$$
\end{problem}
\begin{proof}
	$$
		\dint (2x + 3)^{10}dx = \dint \dfrac{1}{2}(2x + 3)^{10}d(2x +3) = \dfrac{1}{22}y^{11} + C = \dfrac{1}{22}(2x + 3)^{11} + C
	$$
\end{proof}

\begin{problem}(\textbf{Д1659})
	$$
		\dint \dfrac{dx}{(5x - 2)^{\frac{5}{2}}}
	$$
\end{problem}
\begin{proof}
	$$
		\dint \dfrac{dx}{(5x - 2)^{\frac{5}{2}}} = \dint\dfrac{1}{5}\dfrac{dy}{y^{\frac{5}{2}}} = -\dfrac{1}{5}{\cdot}\dfrac{2}{3}{\cdot}y^{-\frac{3}{2}} + C =-\dfrac{2}{15}(5x - 2)^{-\frac{3}{2}} +C
	$$
\end{proof}

\begin{problem}(\textbf{Д1667})
	$$
		\dint\dfrac{dx}{\sin^2{\left(2x + \tfrac{\pi}{4}\right)}}
	$$
\end{problem}
\begin{proof}
	$$
		\dint\dfrac{dx}{\sin^2{\left(2x + \tfrac{\pi}{4}\right)}} = \dfrac{1}{2}\dint\dfrac{dy}{\sin^2(y)} = -\dfrac{1}{2}\ctg{y} + C = -\dfrac{1}{2}\ctg{\left(2x + \tfrac{\pi}{4}\right)} + C
	$$
\end{proof}
\begin{problem}(\textbf{Д1672})
	$$
		\dint\dfrac{dx}{\ch^2{\frac{x}{2}}}
	$$
\end{problem}
\begin{proof}
	$$
		\dint\dfrac{dx}{\ch^2{\left(\frac{x}{2}\right)}} = 2 \dint \dfrac{dy}{\ch^2{y}} = 2 \th{y} + C = 2\th{\left(\frac{x}{2}\right)} + C
	 $$
\end{proof}

Найдем обратную для $\th{x}$ функцию. Заметим, что $\th{x} \in (-1,1)$:
$$
	y = \th{x} = \dfrac{\sh{x}}{\ch{x}} = \dfrac{e^x - e^{-x}}{e^x + e^{-x}} \Rightarrow y(e^x + e^{-x}) = e^x - e^{-x} \Rightarrow y(e^{2x} + 1) = e^{2x} - 1 \Rightarrow
$$
$$
	\Rightarrow ye^{2x} - e^{2x} = -1 - y \Rightarrow e^{2x} = -\dfrac{1 + y}{y - 1} = \dfrac{1+ y}{1 - y} \Rightarrow  x = \th^{-1}{x}= \dfrac{1}{2}\ln{\left( \dfrac{1 + y}{1-y}\right)}, \, |y| < 1
$$
Аналогично для $\cth{x}$. Заметим, что $\cth{x} \in (-\infty, -1) \cup (1, +\infty)$:
$$
	y(e^{2x} - 1) = e^{2x} + 1 \Rightarrow e^{2x} = \dfrac{y + 1}{y - 1} \Rightarrow x = \cth^{-1}{x} = \dfrac{1}{2}\ln{\left( \dfrac{y + 1}{y - 1}\right)}, \, |y| > 1
$$
\newpage
Найдем их производные. Для ареатангенса:
$$
	(\th^{-1}(x))' = \dfrac{1}{2}{\cdot}\dfrac{1}{\frac{1 +y}{1-y}}{\cdot}\dfrac{(1-y) + (1 + y)}{(1 - y)^2} = \dfrac{1}{1 - y^2}
$$
Для ареакотангенса:
$$
	(\cth^{-1}(x))' = \dfrac{1}{2}{\cdot}\dfrac{1}{\frac{y + 1}{y -1}}{\cdot}\dfrac{(y- 1) - (y + 1)}{(y - 1)^2} = - \dfrac{1}{y^2 - 1} = \dfrac{1}{1 - y^2}
$$
\end{document}