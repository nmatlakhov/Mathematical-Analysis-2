\documentclass[12pt]{article}
\usepackage[left=1cm, right=1cm, top=2cm,bottom=1.5cm]{geometry} 

\usepackage[parfill]{parskip}
\usepackage[utf8]{inputenc}
\usepackage[T2A]{fontenc}
\usepackage[russian]{babel}
\usepackage{enumitem}
\usepackage[normalem]{ulem}
\usepackage{amsfonts, amsmath, amsthm, amssymb, mathtools,xcolor}
\usepackage{blkarray}

\usepackage{tabularx}
\usepackage{hhline}

\usepackage{accents}
\usepackage{fancyhdr}
\pagestyle{fancy}
\renewcommand{\headrulewidth}{1.5pt}
\renewcommand{\footrulewidth}{1pt}

\usepackage{graphicx}
\usepackage[figurename=Рис.]{caption}
\usepackage{subcaption}
\usepackage{float}

%%Наименование папки откуда забирать изображения
\graphicspath{ {./images/} }

%%Изменение формата для ввода доказательства
\renewcommand{\proofname}{$\square$  \nopunct}
\renewcommand\qedsymbol{$\blacksquare$}

%%Изменение отступа на таблицах
\addto\captionsrussian{%
	\renewcommand{\proofname}{$\square$ \nopunct}%
}
%% Римские цифры
\newcommand{\RN}[1]{%
	\textup{\uppercase\expandafter{\romannumeral#1}}%
}

%% Для удобства записи
\newcommand{\MR}{\mathbb{R}}
\newcommand{\MC}{\mathbb{C}}
\newcommand{\MQ}{\mathbb{Q}}
\newcommand{\MN}{\mathbb{N}}
\newcommand{\MZ}{\mathbb{Z}}
\newcommand{\MTB}{\mathbb{T}}
\newcommand{\MTI}{\mathbb{I}}
\newcommand{\MI}{\mathrm{I}}
\newcommand{\MCI}{\mathcal{I}}
\newcommand{\MJ}{\mathrm{J}}
\newcommand{\MH}{\mathrm{H}}
\newcommand{\MT}{\mathrm{T}}
\newcommand{\MU}{\mathcal{U}}
\newcommand{\MV}{\mathcal{V}}
\newcommand{\MB}{\mathcal{B}}
\newcommand{\MF}{\mathcal{F}}
\newcommand{\MW}{\mathcal{W}}
\newcommand{\ML}{\mathcal{L}}
\newcommand{\MP}{\mathcal{P}}
\newcommand{\VN}{\varnothing}
\newcommand{\VE}{\varepsilon}
\newcommand{\dx}{\, dx}
\newcommand{\dy}{\, dy}
\newcommand{\dz}{\, dz}
\newcommand{\dd}{\, d}


\theoremstyle{definition}
\newtheorem{defn}{Опр:}
\newtheorem{rem}{Rm:}
\newtheorem{prop}{Утв.}
\newtheorem{exrc}{Упр.}
\newtheorem{problem}{Задача}
\newtheorem{lemma}{Лемма}
\newtheorem{theorem}{Теорема}
\newtheorem{corollary}{Следствие}

\newenvironment{cusdefn}[1]
{\renewcommand\thedefn{#1}\defn}
{\enddefn}

\DeclareRobustCommand{\divby}{%
	\mathrel{\text{\vbox{\baselineskip.65ex\lineskiplimit0pt\hbox{.}\hbox{.}\hbox{.}}}}%
}
\DeclareRobustCommand{\ndivby}{\mkern-1mu\not\mathrel{\mkern4.5mu\divby}\mkern1mu}


%Короткий минус
\DeclareMathSymbol{\SMN}{\mathbin}{AMSa}{"39}
%Длинная шапка
\newcommand{\overbar}[1]{\mkern 1.5mu\overline{\mkern-1.5mu#1\mkern-1.5mu}\mkern 1.5mu}
%Функция знака
\DeclareMathOperator{\sgn}{sgn}

%Функция ранга
\DeclareMathOperator{\rk}{\text{rk}}
\DeclareMathOperator{\diam}{\text{diam}}


%Обозначение константы
\DeclareMathOperator{\const}{\text{const}}

\DeclareMathOperator{\codim}{\text{codim}}

\DeclareMathOperator*{\dsum}{\displaystyle\sum}
\newcommand{\ddsum}[2]{\displaystyle\sum\limits_{#1}^{#2}}

%Интеграл в большом формате
\DeclareMathOperator{\dint}{\displaystyle\int}
\newcommand{\ddint}[2]{\displaystyle\int\limits_{#1}^{#2}}
\newcommand{\ssum}[1]{\displaystyle \sum\limits_{n=1}^{\infty}{#1}_n}

\newcommand{\smallerrel}[1]{\mathrel{\mathpalette\smallerrelaux{#1}}}
\newcommand{\smallerrelaux}[2]{\raisebox{.1ex}{\scalebox{.75}{$#1#2$}}}

\newcommand{\smallin}{\smallerrel{\in}}
\newcommand{\smallnotin}{\smallerrel{\notin}}

\newcommand*{\medcap}{\mathbin{\scalebox{1.25}{\ensuremath{\cap}}}}%
\newcommand*{\medcup}{\mathbin{\scalebox{1.25}{\ensuremath{\cup}}}}%

\makeatletter
\newcommand{\vast}{\bBigg@{3.5}}
\newcommand{\Vast}{\bBigg@{5}}
\makeatother

%Промежуточное значение для sup\inf, поскольку они имеют разную высоту
\newcommand{\newsup}{\mathop{\smash{\mathrm{sup}}}}
\newcommand{\newinf}{\mathop{\mathrm{inf}\vphantom{\mathrm{sup}}}}

%Скалярное произведение
\newcommand{\inner}[2]{\left\langle #1, #2 \right\rangle }
\newcommand{\linsp}[1]{\left\langle #1 \right\rangle }
\newcommand{\linmer}[2]{\left\langle #1 \vert #2\right\rangle }

%Подпись символов снизу
\newcommand{\ubar}[1]{\underaccent{\bar}{#1}}

%% Шапка для букв сверху
\newcommand{\wte}[1]{\widetilde{#1}}
\newcommand{\wht}[1]{\widehat{#1}}
\newcommand{\ovl}[1]{\overline{#1}}

%%Трансформация Фурье
\newcommand{\fourt}[1]{\mathcal{F}\left(#1\right)}
\newcommand{\ifourt}[1]{\mathcal{F}^{-1}\left(#1\right)}

%%Символ вектора
\newcommand{\vecm}[1]{\overrightarrow{#1\,}}

%%Пространстов матриц
\newcommand{\matsq}[1]{\operatorname{Mat}_{#1}}
\newcommand{\mat}[2]{\operatorname{Mat}_{#1, #2}}

%Оператор для действ и мнимых чисел
\DeclareMathOperator{\IM}{\operatorname{Im}}
\DeclareMathOperator{\RE}{\operatorname{Re}}
\DeclareMathOperator{\li}{\operatorname{li}}
\DeclareMathOperator{\GL}{\operatorname{GL}}
\DeclareMathOperator{\SL}{\operatorname{SL}}

%Делимость чисел
\newcommand{\modn}[3]{#1 \equiv #2 \; (\bmod \; #3)}


%%Взятие в скобки, модули и норму
\newcommand{\parfit}[1]{\left( #1 \right)}
\newcommand{\modfit}[1]{\left| #1 \right|}
\newcommand{\sqparfit}[1]{\left\{ #1 \right\}}
\newcommand{\normfit}[1]{\left\| #1 \right\|}

%%Функция для обозначения равномерной сходимости по множеству
\newcommand{\uconv}[1]{\overset{#1}{\rightrightarrows}}
\newcommand{\uconvm}[2]{\overset{#1}{\underset{#2}{\rightrightarrows}}}


%%Функция для обозначения нижнего и верхнего интегралов
\def\upint{\mathchoice%
	{\mkern13mu\overline{\vphantom{\intop}\mkern7mu}\mkern-20mu}%
	{\mkern7mu\overline{\vphantom{\intop}\mkern7mu}\mkern-14mu}%
	{\mkern7mu\overline{\vphantom{\intop}\mkern7mu}\mkern-14mu}%
	{\mkern7mu\overline{\vphantom{\intop}\mkern7mu}\mkern-14mu}%
	\int}
\def\lowint{\mkern3mu\underline{\vphantom{\intop}\mkern7mu}\mkern-10mu\int}

%%След матрицы
\DeclareMathOperator*{\tr}{tr}

\makeatletter
\renewcommand*\env@matrix[1][*\c@MaxMatrixCols c]{%
	\hskip -\arraycolsep
	\let\@ifnextchar\new@ifnextchar
	\array{#1}}
\makeatother


%% Переопределение функции хи, чтобы выглядела более приятно
\makeatletter
\@ifdefinable\@latex@chi{\let\@latex@chi\chi}
\renewcommand*\chi{{\@latex@chi\smash[t]{\mathstrut}}} % want only bottom half of \mathstrut
\makeatletter

\begin{document}
\lhead{Математический анализ - \RN{2}}
\chead{Косухин О.Н.}
\rhead{Семинар - 6: ДЗ}

\textbf{ДЗ}: $1966$ (замена второго типа), $1996$, $2011$ а) и посчитать $\dint \sin^6{x}dx$, $2025$ (тангенс половинного угла), $2029$ (обычный тангенс), $2043$, $2046$.

\section*{Подстановки Эйлера}

\begin{problem}(\textbf{Д1966})
	$$
		\dint \dfrac{dx}{x + \sqrt{x^2 + x + 1 }}
	$$
\end{problem}
\begin{proof}
	См. в семинаре 6.
\end{proof}
\section*{Интегрирование тригонометрических функций}
\subsection*{Использование формул понижения степени}
\begin{problem}(\textbf{Д1996})
	$$
		\dint \sin^5{x}\cos^5{x}dx
	$$
\end{problem}
\begin{proof}
	$$
		\dint \sin^5{x}\cos^5{x}dx = \dint \sin^5{x}\cos^4{x}d(\sin{x}) = \dint t^5(1-t^2)^2dt = \dfrac{t^6}{6} -2\dfrac{t^8}{8} + \dfrac{t^{10}}{10} + C =
	$$
	$$
		= \sin^6{x}{\cdot}\left(\dfrac{1}{6} - \dfrac{1}{4}\sin^2{x} + \dfrac{1}{10}\sin^4{x}\right) + C
	$$
	Либо можно решить через формулы понижения:
	$$
		2\sin{x}\cos{x} = \sin{2x} \Rightarrow \dint \sin^5{x}\cos^5{x}dx = \dfrac{1}{2^5}\dint \sin^5{2x}dx = \dfrac{1}{32}\dint (1 - \cos^2{2x})^2\sin{2x}dx=
	$$
	$$
		= -\dfrac{1}{64}\dint(1 - \cos^2{2x})^2d(\cos{2x}) = -\dfrac{1}{64}\left(\dint dy - 2\dint y^2 dy + \dint y^4 dy\right) =  
	$$
	$$
		=	-\dfrac{\cos{2x}}{64} + \dfrac{\cos^3{2x}}{96} - \dfrac{\cos^5{2x}}{320} + C 
	$$
\end{proof}

\subsection*{Интегрирование по частям}
\begin{problem}(\textbf{Д2011 а)})
	Вывести формулу понижения для интеграла:
	$$
		\MI_n = \dint \sin^n{x}dx
	$$
	и с помощью формулы вычислить:
	$$
		\dint \sin^6{x}dx
	$$
\end{problem}
\begin{proof}
	$$
		\dint \sin^n{x}dx =  - \dint \sin^{n-1}{x}d(\cos{x}) = -\sin^{n-1}{x}\cos{x} + (n-1){\cdot}\dint \sin^{n-2}{x}\cos^2{x}dx =
	$$
	$$
		=-\sin^{n-1}{x}\cos{x} + (n-1)(\MI_{n-2} - \MI_{n}) \Rightarrow n\MI_n = -\sin^{n-1}{x}\cos{x} + (n-1)\MI_{n-2}  \Rightarrow
	$$
	$$
		\Rightarrow \MI_n = -\dfrac{1}{n}\sin^{n-1}{x}\cos{x} + \dfrac{(n-1)}{n}\MI_{n-2}
	$$
	Посчитаем конкретный пример при $n = 6$:
	$$
		\MI_6 = -\dfrac{1}{6}\sin^{5}{x}\cos{x} + \dfrac{5}{6}{\cdot}\left(-\dfrac{1}{4}\sin^{3}{x}\cos{x} + \dfrac{3}{4}{\cdot}\dint\sin^2{x}dx\right)
	$$
	$$
		\dint \sin^2{x}dx = \dfrac{1}{2}\dint (1 - \cos{2x})dx = \dfrac{x}{2} - \dfrac{1}{4}\sin{2x} + C
	$$
\end{proof}

\subsection*{Интегрирование рациональных тригонометрических функций}
\begin{problem}(\textbf{Д2025})
	$$
		\dint \dfrac{dx}{2 \sin{x} - \cos{x} + 5}
	$$
\end{problem}
\begin{proof}
	Ни один из описанных случаев нам не подходит, поэтому воспользуемся заменой тангенса половинного угла:
	$$
		t = \tg{\tfrac{x}{2}} \Rightarrow \sin{x} = \dfrac{2t}{1 + t^2}, \, \cos{x} = \dfrac{1 -t^2}{1 + t^2}, \, dx = \dfrac{2dt}{1 + t^2}
	$$
	Это верно, поскольку:
	$$
		\sin{x} = 2\sin{\tfrac{x}{2}}\cos{\tfrac{x}{2}} = 2 \tg{\tfrac{x}{2}}\cos^2{\tfrac{x}{2}} = 2 \tg{\tfrac{x}{2}}\dfrac{1}{1 + \tg^2{\tfrac{x}{2}}} = \dfrac{2t}{1 + t^2}
	$$
	$$
		\cos{x} = 2\cos^2{\tfrac{x}{2}} - 1 = 2 \dfrac{1}{1 + \tg^2{\tfrac{x}{2}}} - 1 = \dfrac{1 - t^2}{1 + t^2}
	$$
	$$
		t = \tg{\tfrac{x}{2}} \Rightarrow x = 2\arctg{t} \Rightarrow dx = \dfrac{2dt}{1 + t^2}
	$$
	Таким образом:
	$$
		\dint \dfrac{dx}{2 \sin{x} - \cos{x} + 5} = \dint \dfrac{2dt}{1 + t^2}{\cdot}\dfrac{1}{\tfrac{4t}{1 + t^2} - \tfrac{1 - t^2}{1+ t^2} + 5} = \dint \dfrac{2dt}{4t - 1 + t^2 + 5 + 5t^2} = 
	$$
	$$
		=	\dint \dfrac{2dt}{6t^2  + 4t + 4} = \dint \dfrac{dt}{3t^2 + 2t + 2} = 
		\dfrac{1}{3}\dint\dfrac{dt}{t^2 + \tfrac{2}{3}t + \tfrac{1}{9} + \tfrac{5}{9}} = \dfrac{1}{3}\dint \dfrac{d\left(t + \tfrac{1}{3}\right)}{\left(t + \tfrac{1}{3}\right)^2 + \tfrac{5}{9}} =
	$$
	$$
		= \dfrac{1}{\sqrt{5}}\arctg{\left(\dfrac{3t + 1}{\sqrt{5}}\right)} + C = \dfrac{1}{\sqrt{5}}\arctg{\left(\dfrac{3\tg{\tfrac{x}{2}} + 1}{\sqrt{5}}\right)} + C
	$$
\end{proof}
\newpage
\begin{problem}(\textbf{Д2029})
	$$
		\dint \dfrac{\sin^2{x}}{1 + \sin^2{x}}dx
	$$
\end{problem}
\begin{proof}
	Здесь подходят замены второго типа: $t = \tg{x}$, тогда:
	$$
		x = \arctg{t}, \, dx = \dfrac{1}{1 + t^2}dt, \, \sin^2{x} = \dfrac{t^2}{1 +t^2} \Rightarrow \dint \dfrac{\sin^2{x}}{1 + \sin^2{x}}dx= \dint dx - \dint \dfrac{dx}{1 + \sin^2{x}} =
	$$
	$$
		= x - \dint \dfrac{dt}{(1 + t^2){\cdot}\left(1 + \tfrac{t^2}{1+t^2}\right)} = x - \dint\dfrac{dt}{1 + 2t^2} = x - \dfrac{1}{2}\dint\dfrac{dt}{t^2 + \tfrac{1}{2}} = x - \dfrac{1}{2}{\cdot}\sqrt{2}{\cdot}\arctg{\left(\sqrt{2}t\right)} + C =
	$$
	$$
		= x -\dfrac{1}{\sqrt{2}}\arctg{\left(\sqrt{2}\tg{x}\right)} + C
	$$
\end{proof}

\begin{problem}(\textbf{Д2043})
	$$
		\dint \dfrac{\sin{x} - \cos{x}}{\sin{x} + 2\cos{x}}dx
	$$
\end{problem}
\begin{proof}
	Применим формулу из задачи $2042$:
	$$
		\dint \dfrac{a_1 \sin{x} + b_1 \cos{x}}{a \sin{x} + b\cos{x}}dx = \dint \dfrac{pf(x) + qf'(x)}{f(x)}dx = px  + q\ln{|f(x)|} + C
	$$
	$$
		f(x) = 1{\cdot}\sin{x} + 2{\cdot}\cos{x}, \, f'(x) = 1{\cdot}\cos{x} - 2{\cdot}\sin{x} \Rightarrow 
	$$
	$$
		\Rightarrow p\sin{x} + 2p\cos{x} + q\cos{x} - 2q\sin{x} = \sin{x} - \cos{x} \Rightarrow p -2q = 1, \, 2p + q = -1 \Rightarrow 
	$$	
	$$	
		\Rightarrow p = -\dfrac{1}{5}, \, q = -\dfrac{3}{5} \Rightarrow
		\dint \dfrac{\sin{x} - \cos{x}}{\sin{x} + 2\cos{x}}dx = -\dfrac{x}{5} - \dfrac{3}{5}\ln{|\sin{x} + 2\cos{x}|} + C
	$$
\end{proof}

\begin{problem}(\textbf{Д2046})
	Доказать, что:
	$$
		\dint \dfrac{a_1 \sin{x} + b_1 \cos{x} + c_1}{a\sin{x} + b\cos{x} + c}dx = Ax + B\ln{|a\sin{x} + b\cos{x} + c}| + C\dint\dfrac{dx}{a\sin{x} + b\cos{x} + c}
	$$
	где $A, B,C$ - некоторые постоянные коэффициенты
\end{problem}
\begin{proof}
	По аналогии с задачей $2042$, введём обозначения:
	$$
		f(x) = a\sin{x} + b\cos{x} + c, \, f'(x) = -b\sin{x} + a\cos{x}
	$$
	Векторы $f(x), f'(x)$ и $1 \in \MR$ - линейно независимы и образуют базис:
	$$
		\begin{vmatrix}
			a & b & 0 \\
			-b & a & 0 \\
			0 & 0 & 1
		\end{vmatrix} = a^2 + 0 + 0 - 0 - (-b)b - 0 = a^2 + b^2\Rightarrow a \neq 0 \vee b \neq 0 \Rightarrow a^2 + b^2 > 0
	$$
	Тогда, вектор написанный сверху можно выразить через $f(x), f'(x)$ и $1$ с некоторыми коэффициентами:
	$$
		\dint \dfrac{a_1 \sin{x} + b_1 \cos{x} + c_1}{a\sin{x} + b\cos{x} + c}dx = \dint \dfrac{A{\cdot}f(x) + B{\cdot}f'(x) + C{\cdot}1}{f(x)}dx = A\dint dx + B\dint \dfrac{d(f(x))}{f(x)} + C\dint\dfrac{dx}{f(x)} = 
	$$
	$$
		= Ax + B\ln{|f(x)|} + C\dint{dx}{f(x)} = Ax + B\ln{|a\sin{x} + b\cos{x} + c|} + C \dint \dfrac{dx}{a\sin{x} + b\cos{x} + c}
	$$
\end{proof}


\end{document}