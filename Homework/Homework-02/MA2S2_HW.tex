\documentclass[12pt]{article}
\usepackage[left=1cm, right=1cm, top=2cm,bottom=1.5cm]{geometry} 

\usepackage[parfill]{parskip}
\usepackage[utf8]{inputenc}
\usepackage[T2A]{fontenc}
\usepackage[russian]{babel}
\usepackage{enumitem}
\usepackage[normalem]{ulem}
\usepackage{amsfonts, amsmath, amsthm, amssymb, mathtools,xcolor}
\usepackage{blkarray}

\usepackage{tabularx}
\usepackage{hhline}

\usepackage{accents}
\usepackage{fancyhdr}
\pagestyle{fancy}
\renewcommand{\headrulewidth}{1.5pt}
\renewcommand{\footrulewidth}{1pt}

\usepackage{graphicx}
\usepackage[figurename=Рис.]{caption}
\usepackage{subcaption}
\usepackage{float}

%%Наименование папки откуда забирать изображения
\graphicspath{ {./images/} }

%%Изменение формата для ввода доказательства
\renewcommand{\proofname}{$\square$  \nopunct}
\renewcommand\qedsymbol{$\blacksquare$}

%%Изменение отступа на таблицах
\addto\captionsrussian{%
	\renewcommand{\proofname}{$\square$ \nopunct}%
}
%% Римские цифры
\newcommand{\RN}[1]{%
	\textup{\uppercase\expandafter{\romannumeral#1}}%
}

%% Для удобства записи
\newcommand{\MR}{\mathbb{R}}
\newcommand{\MC}{\mathbb{C}}
\newcommand{\MQ}{\mathbb{Q}}
\newcommand{\MN}{\mathbb{N}}
\newcommand{\MZ}{\mathbb{Z}}
\newcommand{\MTB}{\mathbb{T}}
\newcommand{\MTI}{\mathbb{I}}
\newcommand{\MI}{\mathrm{I}}
\newcommand{\MCI}{\mathcal{I}}
\newcommand{\MJ}{\mathrm{J}}
\newcommand{\MH}{\mathrm{H}}
\newcommand{\MT}{\mathrm{T}}
\newcommand{\MU}{\mathcal{U}}
\newcommand{\MV}{\mathcal{V}}
\newcommand{\MB}{\mathcal{B}}
\newcommand{\MF}{\mathcal{F}}
\newcommand{\MW}{\mathcal{W}}
\newcommand{\ML}{\mathcal{L}}
\newcommand{\MP}{\mathcal{P}}
\newcommand{\VN}{\varnothing}
\newcommand{\VE}{\varepsilon}

\theoremstyle{definition}
\newtheorem{defn}{Опр:}
\newtheorem{rem}{Rm:}
\newtheorem{prop}{Утв.}
\newtheorem{exrc}{Упр.}
\newtheorem{problem}{Задача}
\newtheorem{lemma}{Лемма}
\newtheorem{theorem}{Теорема}
\newtheorem{corollary}{Следствие}

\newenvironment{cusdefn}[1]
{\renewcommand\thedefn{#1}\defn}
{\enddefn}

\DeclareRobustCommand{\divby}{%
	\mathrel{\text{\vbox{\baselineskip.65ex\lineskiplimit0pt\hbox{.}\hbox{.}\hbox{.}}}}%
}
%Короткий минус
\DeclareMathSymbol{\SMN}{\mathbin}{AMSa}{"39}
%Длинная шапка
\newcommand{\overbar}[1]{\mkern 1.5mu\overline{\mkern-1.5mu#1\mkern-1.5mu}\mkern 1.5mu}
%Функция знака
\DeclareMathOperator{\sgn}{sgn}

%Функция ранга
\DeclareMathOperator{\rk}{\text{rk}}
\DeclareMathOperator{\diam}{\text{diam}}


%Обозначение константы
\DeclareMathOperator{\const}{\text{const}}

\DeclareMathOperator{\codim}{\text{codim}}

\DeclareMathOperator*{\dsum}{\displaystyle\sum}
\newcommand{\ddsum}[2]{\displaystyle\sum\limits_{#1}^{#2}}

%Интеграл в большом формате
\DeclareMathOperator{\dint}{\displaystyle\int}
\newcommand{\ddint}[2]{\displaystyle\int\limits_{#1}^{#2}}
\newcommand{\ssum}[1]{\displaystyle \sum\limits_{n=1}^{\infty}{#1}_n}

\newcommand{\smallerrel}[1]{\mathrel{\mathpalette\smallerrelaux{#1}}}
\newcommand{\smallerrelaux}[2]{\raisebox{.1ex}{\scalebox{.75}{$#1#2$}}}

\newcommand{\smallin}{\smallerrel{\in}}
\newcommand{\smallnotin}{\smallerrel{\notin}}

\newcommand*{\medcap}{\mathbin{\scalebox{1.25}{\ensuremath{\cap}}}}%
\newcommand*{\medcup}{\mathbin{\scalebox{1.25}{\ensuremath{\cup}}}}%

\makeatletter
\newcommand{\vast}{\bBigg@{3.5}}
\newcommand{\Vast}{\bBigg@{5}}
\makeatother

%Промежуточное значение для sup\inf, поскольку они имеют разную высоту
\newcommand{\newsup}{\mathop{\smash{\mathrm{sup}}}}
\newcommand{\newinf}{\mathop{\mathrm{inf}\vphantom{\mathrm{sup}}}}

%Скалярное произведение
\newcommand{\inner}[2]{\left\langle #1, #2 \right\rangle }
\newcommand{\linsp}[1]{\left\langle #1 \right\rangle }
\newcommand{\linmer}[2]{\left\langle #1 \vert #2\right\rangle }

%Подпись символов снизу
\newcommand{\ubar}[1]{\underaccent{\bar}{#1}}

%% Шапка для букв сверху
\newcommand{\wte}[1]{\widetilde{#1}}
\newcommand{\wht}[1]{\widehat{#1}}

%%Трансформация Фурье
\newcommand{\fourt}[1]{\mathcal{F}\left(#1\right)}
\newcommand{\ifourt}[1]{\mathcal{F}^{-1}\left(#1\right)}

%%Символ вектора
\newcommand{\vecm}[1]{\overrightarrow{#1\,}}

%%Пространстов матриц
\newcommand{\mat}[2]{\operatorname{Mat}_{#1\times #2}}


%%Взятие в скобки, модули и норму
\newcommand{\parfit}[1]{\left( #1 \right)}
\newcommand{\modfit}[1]{\left| #1 \right|}
\newcommand{\sqparfit}[1]{\left\{ #1 \right\}}
\newcommand{\normfit}[1]{\left\| #1 \right\|}

%%Функция для обозначения равномерной сходимости по множеству
\newcommand{\uconv}[1]{\overset{#1}{\rightrightarrows}}
\newcommand{\uconvm}[2]{\overset{#1}{\underset{#2}{\rightrightarrows}}}


%%Функция для обозначения нижнего и верхнего интегралов
\def\upint{\mathchoice%
	{\mkern13mu\overline{\vphantom{\intop}\mkern7mu}\mkern-20mu}%
	{\mkern7mu\overline{\vphantom{\intop}\mkern7mu}\mkern-14mu}%
	{\mkern7mu\overline{\vphantom{\intop}\mkern7mu}\mkern-14mu}%
	{\mkern7mu\overline{\vphantom{\intop}\mkern7mu}\mkern-14mu}%
	\int}
\def\lowint{\mkern3mu\underline{\vphantom{\intop}\mkern7mu}\mkern-10mu\int}

%%След матрицы
\DeclareMathOperator*{\tr}{tr}

\makeatletter
\renewcommand*\env@matrix[1][*\c@MaxMatrixCols c]{%
	\hskip -\arraycolsep
	\let\@ifnextchar\new@ifnextchar
	\array{#1}}
\makeatother


%% Переопределение функции хи, чтобы выглядела более приятно
\makeatletter
\@ifdefinable\@latex@chi{\let\@latex@chi\chi}
\renewcommand*\chi{{\@latex@chi\smash[t]{\mathstrut}}} % want only bottom half of \mathstrut
\makeatletter

\begin{document}
\lhead{Математический анализ - \RN{2}}
\chead{Косухин О.Н.}
\rhead{Семинар - 2: ДЗ}
\section*{Неопределенный интеграл}
\subsection*{Замена переменных и интегрирование по частям}
\textbf{ДЗ}: $1690, \, 1695, \, 1703$ (перейти к половинному углу: $\sin{x} = 2\sin{\frac{x}{2}}\cos{\frac{x}{2}}$, замена $t= \tg{\frac{x}{2}}$). 

\textbf{ДЗ}: подстановки: $1777, \, 1780, \, 1785, \, 1790, \, 1829$.

\begin{problem}(\textbf{Д1690})
	$$
		\dint\dfrac{e^x dx}{2 + e^x}
	$$
\end{problem}
\begin{proof}
	$$
		\dint\dfrac{e^x dx}{2 + e^x} = \dint \dfrac{du}{u} = \ln{u} + C = \ln{(2 + e^x)} + C
	$$
\end{proof}
\begin{problem}(\textbf{Д1695})
	$$
		\dint \sin^5{x}\cos{x}dx 
	$$
\end{problem}
\begin{proof}
	$$
		\dint \sin^5{x}\cos{x}dx  = \dint u^5 du = \dfrac{u^6}{6} + C = \dfrac{\sin^6{x}}{6} + C
	$$
\end{proof}
\begin{problem}(\textbf{Д1703})
	$$
		\dint \dfrac{dx}{\sin{x}}
	$$
\end{problem}
\begin{proof}
	$$
		\dint \dfrac{dx}{\sin{x}} = \dint \dfrac{dx}{2\sin{\frac{x}{2}}\cos{\frac{x}{2}}} = \dint \dfrac{du}{\sin{u}\cos{u}} = \dint\dfrac{du}{\tg{u}\cos^2{u}} = \dint\dfrac{d(\tg{u})}{\tg{u}} = \ln{\left|\tg{u}\right|} + C = \ln{\left|\tg{\frac{x}{2}}\right|} + C 
	$$
\end{proof}

\begin{problem}(\textbf{Д1777})
	$$
		\dint\dfrac{\arctg{\sqrt{x}}}{\sqrt{x}}{\cdot}\dfrac{dx}{1+x}
	$$
\end{problem}
\begin{proof}
	$$
		\dint\dfrac{\arctg{\sqrt{x}}}{\sqrt{x}}{\cdot}\dfrac{dx}{1+x} = \dint \arctg{\sqrt{x}}{\cdot}\dfrac{2d(\sqrt{x})}{1 + x} = 2\dint \arctg{u} \dfrac{du}{1 + u^2} = 2 \dint w dw = w^2 + C = (\arctg{\sqrt{x}})^2 + C
	$$
\end{proof}
\newpage
\begin{problem}(\textbf{Д1780})
	$$
		\dint\sqrt{1 - x^2}dx
	$$
\end{problem}
\begin{proof}
	$$
		\dint\sqrt{1 - x^2}dx = |x  = \sin{t}| = \dint \sqrt{\cos^2{t}}\cos{t}dt
	$$
	Пусть $t \in\left[-\frac{\pi}{2},\frac{\pi}{2}\right]$, тогда $|\cos{t}| = \cos{t}$ и $\sin{t} \in [-1,1]$. Следовательно мы получим:
	$$
		\dint \sqrt{\cos^2{t}}\cos{t}dt = \dint \cos^2{t}dt = \dint \left(\dfrac{ 1 + \cos{(2t)}}{2}\right)dt = \dfrac{t}{2} + \dfrac{1}{4}\dint \cos{u}du =
	$$
	$$
		= \dfrac{t}{2}	+ \dfrac{\sin{(2t)}}{4} + C= \dfrac{\arcsin{x}}{2} + \dfrac{2\sin{t}\cos{t}}{4} + C =  \dfrac{\arcsin{x}}{2} + \sqrt{1 -x^2}{\cdot}\dfrac{x}{2} + C
	$$
\end{proof}
\begin{problem}(\textbf{Д1785})
	$$
		\dint\sqrt{(x-a)(b-x)}dx
	$$
\end{problem}
\begin{proof}
	Сделаем замену: $x - a = (b - a){\cdot}\sin^2{t}$.
	$$
		b - x = (b - a) - (x - a) = (b - a) - (b - a){\cdot}\sin^2{t} = (b-a){\cdot}\cos^2{t}
	$$
	Без потери общности будем считать, что $a < b$. Поскольку наш трехчлен имеет вид: $(x - a)(b - x)$, то он положителен только на интервале $(a,b) \Rightarrow $ подразумевается, что мы работаем на этом интервале. 
	Выберем значения $t \in \left(0,\frac{\pi}{2}\right)$, тогда: 
	$$
		t \in \left(0,\frac{\pi}{2}\right) \Rightarrow \sin^2{t} \in (0,1) \Rightarrow (b-a)\sin^2{t} \in (0, b-a) \Rightarrow x = a + (b-a)\sin^2{t} \in (a,b)
	$$
	Таким образом, из замены мы получим:
	$$
		x - a = (b - a){\cdot}\sin^2{t} \Rightarrow t = \arcsin{\left(\sqrt{\dfrac{x- a}{b-a}}\right)}, \quad dx = 2(b-a)\sin{t}\cos{t}dt
	$$
	$$
		\dint \sqrt{(x-a)(b-x)}dx = \dint (b-a){\cdot}\sin{t}{\cdot}\cos{t} {\cdot}2(b-a){\cdot}\sin{t}{\cdot}\cos{t}dt = 2(b-a)^2\dint \sin^2{t}\cos^2{t}dt=
	$$
	$$
		=\dfrac{(b-a)^2}{2}\dint \sin^{2}(2t)dt = \dfrac{(b-a)^2}{2}\dint \dfrac{1 - \cos{(4t)}}{2}dt = \dfrac{(b-a)^2}{4}t - \dfrac{(b-a)^2}{4{\cdot}4}\sin{(4t)} +C 
	$$
	Поскольку $t \in (0,\frac{\pi}{2})$, то $\cos{t} > 0$ и будет верно равенство:
	$$
		\sin{(4t)} = 2\sin{(2t)}{\cdot}\cos{(2t)} = 2\sin{(2t)}{\cdot}(1 - 2\sin^2{t}) = 4\sin{t}{\cdot}\cos{t}{\cdot}(1 - 2\sin^2{t}) =
	$$
	$$
		= 4\sqrt{\dfrac{x -a}{b-a}}{\cdot}\sqrt{1 - \dfrac{x -a}{b-a}}{\cdot}\left(\dfrac{b-a - 2x + 2a}{b-a}\right) = 4 \sqrt{(x-a)(b-x)}{\cdot}\dfrac{(a + b) - 2x}{(b-a)^2}
	$$

	$$
		\dint\sqrt{(x-a)(b-x)}dx = \dfrac{(b-a)^2}{4}\arcsin{\left(\sqrt{\dfrac{x- a}{b-a}}\right)} + \dfrac{2x - (a+ b)}{4}{\cdot}\sqrt{(x-a)(b-x)} + C
	$$
\end{proof}
\newpage
\begin{problem}(\textbf{Д1790})
	$$
		\dint \sqrt{(x+a)(x+b)}dx,\quad x + a = (b-a)\sh^2{t}
	$$
\end{problem}
\begin{proof}
	Под корнем у нас находится квадратный трехчлен с корнями $-a$ и $-b$, чтобы корень выражения был определен нам необходимо верность неравенства: $(x+a)(x+b) \geq  0$. Без ограничения общности, пусть $-a < -b \Rightarrow x \in (-\infty,-a)\cup (-b, + \infty)$. 	
	Сделаем замену: 
	$$
		x+ a = (b-a)\sh^2{t} \Rightarrow \sh^2{t} = \dfrac{x + a}{b-a} \Rightarrow t =\sh^{-1}\left(\sqrt{\dfrac{x+a}{b-a}}\right) = \ln{\left(\sqrt{\dfrac{x+a}{b-a}} + \sqrt{\dfrac{x+b}{b-a}}\right)}
	$$ 
	Будем выбирать $t \geq 0 \Rightarrow \sh{t} > 0, \, \ch{t} > 0$. Мы уже знаем, что $-a < -b \Rightarrow a >  b \Rightarrow x + a < 0$, тогда выражение под корнем будет положительным. На промежутке $(-b,+\infty)$ надо просто поменять $b$ и $a$ местами. Заметим, что:
	$$
		x + b = x + a + (b - a) = (b-a)\sh^2{t} + (b-a) = (b-a){\cdot}(\sh^2{t} + 1) = (b-a){\cdot}\ch^2{t} \Rightarrow
	$$
	$$
		\Rightarrow \sqrt{(x+a)(x+b)} = \sqrt{(b-a)^2\sh^2{t}\sh^2{t}} = -(b-a)\sh{t}\ch{t}, \quad dx = d(x + a) = (b-a)2\sh{t}\ch{t}dt \Rightarrow
	$$
	$$
		\Rightarrow  \dint \sqrt{(x+a)(x+b)}dx = -2\dint (b-a)^2 \sh^2{t}\ch^2{t}dt
	$$
	Воспользуемся здесь формулой: $\sh{(2t)} = 2\sh{t}\ch{t}$, тогда:
	$$
		-2\dint (b-a)^2 \sh^2{t}\ch^2{t}dt = -\dfrac{(b-a)^2}{2}\dint \sh^2{(2t)}dt 
	$$
	$$
		\ch{(4t)} = \dfrac{e^{4t} + e^{-4t}}{2}, \, \sh{(2t)} = \dfrac{e^{2t} - e^{-2t}}{2} \Rightarrow \sh^2{(2t)} = \dfrac{e^{4t} -2 + e^{-4t}}{4} = \dfrac{\ch{(4t)} - 1}{2} \Rightarrow
	$$
	$$
		\Rightarrow -\dfrac{(b-a)^2}{2}\dint \sh^2{(2t)}dt = -\dfrac{(b-a)^2}{4}\dint \ch{(4t)} - 1 dt = -\dfrac{(b-a)^2}{4}{\cdot}\left(\dfrac{1}{4}\sh{(4t)} - t\right) + C =
	$$
	$$
		=	\dfrac{(b-a)^2}{4}{\cdot}\left(-\dfrac{1}{2}\sh{(2t)}\ch{(2t)} + t\right) + C = \dfrac{(b-a)^2}{4}{\cdot}\left(-\sh{t}\ch{t}\ch{(2t)} + t\right) + C
	$$
	$$
		\ch{(2t)} = \dfrac{e^{2t} + e^{-2t}}{2} = 2\sh^2{t} + 1  \Rightarrow
	$$
	$$
		\dfrac{(b-a)^2}{4}{\cdot}\left(-\sh{t}\ch{t}{\cdot}\ch{(2t)} + t\right) + C =	\dfrac{(b-a)^2}{4}\left(-\sqrt{\dfrac{x+a}{b-a}}{\cdot}\sqrt{\dfrac{x + b}{b-a}}{\cdot}\left( 2{\cdot}\dfrac{x+a}{b-a} + 1\right) +t \right) +C = 
	$$
	$$
		=	\dfrac{\sqrt{(x+a)(x+b)}}{4}{\cdot}(2x + a + b) + \dfrac{(b-a)^2}{4}\ln{\left(\sqrt{\dfrac{x+a}{b-a}} + \sqrt{\dfrac{x+b}{b-a}}\right)} +C
	$$
\end{proof}
\newpage
\begin{problem}(\textbf{Д1829})
	$$
		\dint e^{a x}\sin{(bx)}dx
	$$
\end{problem}
\begin{proof}
	$$
		\MI = \dint e^{a x}\sin{(bx)}dx = \dfrac{1}{a}e^{a x}\sin{(bx)} - \dfrac{b}{a}\dint e^{a x} \cos{(bx)}dx = 
	$$
	$$	
		= \dfrac{1}{a}e^{a x}\sin{(bx)} - \dfrac{b}{a^2}e^{a x} \cos{(bx)} - \dfrac{b^2}{a^2}\dint e^{a x}\sin{(bx)}dx \Rightarrow
	$$
	$$
		\Rightarrow \left(1 + \dfrac{b^2}{a^2}\right)\MI = \dfrac{1}{a}e^{\alpha x}\sin{(bx)} - \dfrac{b}{a^2}e^{a x} \cos{(bx)} +C \Rightarrow \MI = \dfrac{a}{a^2 + b^2}e^{a x}\sin{(bx)} - \dfrac{b}{a^2 + b^2}e^{a x}\cos{(bx)} + C
	$$
\end{proof}
		
\end{document}