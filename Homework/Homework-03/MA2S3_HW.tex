\documentclass[12pt]{article}
\usepackage[left=1cm, right=1cm, top=2cm,bottom=1.5cm]{geometry} 

\usepackage[parfill]{parskip}
\usepackage[utf8]{inputenc}
\usepackage[T2A]{fontenc}
\usepackage[russian]{babel}
\usepackage{enumitem}
\usepackage[normalem]{ulem}
\usepackage{amsfonts, amsmath, amsthm, amssymb, mathtools,xcolor}
\usepackage{blkarray}

\usepackage{tabularx}
\usepackage{hhline}

\usepackage{accents}
\usepackage{fancyhdr}
\pagestyle{fancy}
\renewcommand{\headrulewidth}{1.5pt}
\renewcommand{\footrulewidth}{1pt}

\usepackage{graphicx}
\usepackage[figurename=Рис.]{caption}
\usepackage{subcaption}
\usepackage{float}

%%Наименование папки откуда забирать изображения
\graphicspath{ {./images/} }

%%Изменение формата для ввода доказательства
\renewcommand{\proofname}{$\square$  \nopunct}
\renewcommand\qedsymbol{$\blacksquare$}

%%Изменение отступа на таблицах
\addto\captionsrussian{%
	\renewcommand{\proofname}{$\square$ \nopunct}%
}
%% Римские цифры
\newcommand{\RN}[1]{%
	\textup{\uppercase\expandafter{\romannumeral#1}}%
}

%% Для удобства записи
\newcommand{\MR}{\mathbb{R}}
\newcommand{\MC}{\mathbb{C}}
\newcommand{\MQ}{\mathbb{Q}}
\newcommand{\MN}{\mathbb{N}}
\newcommand{\MZ}{\mathbb{Z}}
\newcommand{\MTB}{\mathbb{T}}
\newcommand{\MTI}{\mathbb{I}}
\newcommand{\MI}{\mathrm{I}}
\newcommand{\MCI}{\mathcal{I}}
\newcommand{\MJ}{\mathrm{J}}
\newcommand{\MH}{\mathrm{H}}
\newcommand{\MT}{\mathrm{T}}
\newcommand{\MU}{\mathcal{U}}
\newcommand{\MV}{\mathcal{V}}
\newcommand{\MB}{\mathcal{B}}
\newcommand{\MF}{\mathcal{F}}
\newcommand{\MW}{\mathcal{W}}
\newcommand{\ML}{\mathcal{L}}
\newcommand{\MP}{\mathcal{P}}
\newcommand{\VN}{\varnothing}
\newcommand{\VE}{\varepsilon}

\theoremstyle{definition}
\newtheorem{defn}{Опр:}
\newtheorem{rem}{Rm:}
\newtheorem{prop}{Утв.}
\newtheorem{exrc}{Упр.}
\newtheorem{problem}{Задача}
\newtheorem{lemma}{Лемма}
\newtheorem{theorem}{Теорема}
\newtheorem{corollary}{Следствие}

\newenvironment{cusdefn}[1]
{\renewcommand\thedefn{#1}\defn}
{\enddefn}

\DeclareRobustCommand{\divby}{%
	\mathrel{\text{\vbox{\baselineskip.65ex\lineskiplimit0pt\hbox{.}\hbox{.}\hbox{.}}}}%
}
%Короткий минус
\DeclareMathSymbol{\SMN}{\mathbin}{AMSa}{"39}
%Длинная шапка
\newcommand{\overbar}[1]{\mkern 1.5mu\overline{\mkern-1.5mu#1\mkern-1.5mu}\mkern 1.5mu}
%Функция знака
\DeclareMathOperator{\sgn}{sgn}

%Функция ранга
\DeclareMathOperator{\rk}{\text{rk}}
\DeclareMathOperator{\diam}{\text{diam}}


%Обозначение константы
\DeclareMathOperator{\const}{\text{const}}

\DeclareMathOperator{\codim}{\text{codim}}

\DeclareMathOperator*{\dsum}{\displaystyle\sum}
\newcommand{\ddsum}[2]{\displaystyle\sum\limits_{#1}^{#2}}

%Интеграл в большом формате
\DeclareMathOperator{\dint}{\displaystyle\int}
\newcommand{\ddint}[2]{\displaystyle\int\limits_{#1}^{#2}}
\newcommand{\ssum}[1]{\displaystyle \sum\limits_{n=1}^{\infty}{#1}_n}

\newcommand{\smallerrel}[1]{\mathrel{\mathpalette\smallerrelaux{#1}}}
\newcommand{\smallerrelaux}[2]{\raisebox{.1ex}{\scalebox{.75}{$#1#2$}}}

\newcommand{\smallin}{\smallerrel{\in}}
\newcommand{\smallnotin}{\smallerrel{\notin}}

\newcommand*{\medcap}{\mathbin{\scalebox{1.25}{\ensuremath{\cap}}}}%
\newcommand*{\medcup}{\mathbin{\scalebox{1.25}{\ensuremath{\cup}}}}%

\makeatletter
\newcommand{\vast}{\bBigg@{3.5}}
\newcommand{\Vast}{\bBigg@{5}}
\makeatother

%Промежуточное значение для sup\inf, поскольку они имеют разную высоту
\newcommand{\newsup}{\mathop{\smash{\mathrm{sup}}}}
\newcommand{\newinf}{\mathop{\mathrm{inf}\vphantom{\mathrm{sup}}}}

%Скалярное произведение
\newcommand{\inner}[2]{\left\langle #1, #2 \right\rangle }
\newcommand{\linsp}[1]{\left\langle #1 \right\rangle }
\newcommand{\linmer}[2]{\left\langle #1 \vert #2\right\rangle }

%Подпись символов снизу
\newcommand{\ubar}[1]{\underaccent{\bar}{#1}}

%% Шапка для букв сверху
\newcommand{\wte}[1]{\widetilde{#1}}
\newcommand{\wht}[1]{\widehat{#1}}

%%Трансформация Фурье
\newcommand{\fourt}[1]{\mathcal{F}\left(#1\right)}
\newcommand{\ifourt}[1]{\mathcal{F}^{-1}\left(#1\right)}

%%Символ вектора
\newcommand{\vecm}[1]{\overrightarrow{#1\,}}

%%Пространстов матриц
\newcommand{\mat}[2]{\operatorname{Mat}_{#1\times #2}}


%%Взятие в скобки, модули и норму
\newcommand{\parfit}[1]{\left( #1 \right)}
\newcommand{\modfit}[1]{\left| #1 \right|}
\newcommand{\sqparfit}[1]{\left\{ #1 \right\}}
\newcommand{\normfit}[1]{\left\| #1 \right\|}

%%Функция для обозначения равномерной сходимости по множеству
\newcommand{\uconv}[1]{\overset{#1}{\rightrightarrows}}
\newcommand{\uconvm}[2]{\overset{#1}{\underset{#2}{\rightrightarrows}}}


%%Функция для обозначения нижнего и верхнего интегралов
\def\upint{\mathchoice%
	{\mkern13mu\overline{\vphantom{\intop}\mkern7mu}\mkern-20mu}%
	{\mkern7mu\overline{\vphantom{\intop}\mkern7mu}\mkern-14mu}%
	{\mkern7mu\overline{\vphantom{\intop}\mkern7mu}\mkern-14mu}%
	{\mkern7mu\overline{\vphantom{\intop}\mkern7mu}\mkern-14mu}%
	\int}
\def\lowint{\mkern3mu\underline{\vphantom{\intop}\mkern7mu}\mkern-10mu\int}

%%След матрицы
\DeclareMathOperator*{\tr}{tr}

\makeatletter
\renewcommand*\env@matrix[1][*\c@MaxMatrixCols c]{%
	\hskip -\arraycolsep
	\let\@ifnextchar\new@ifnextchar
	\array{#1}}
\makeatother


%% Переопределение функции хи, чтобы выглядела более приятно
\makeatletter
\@ifdefinable\@latex@chi{\let\@latex@chi\chi}
\renewcommand*\chi{{\@latex@chi\smash[t]{\mathstrut}}} % want only bottom half of \mathstrut
\makeatletter

\begin{document}
\lhead{Математический анализ - \RN{2}}
\chead{Косухин О.Н.}
\rhead{Семинар - 3: ДЗ}
\section*{Неопределенный интеграл}
\textbf{ДЗ}: $1808, \, 1822, \, 1837, \, 1842, \, 1867, \, 1870$ (можно разложить квадратный трехчлен), $\, 1882$.

\begin{problem}(\textbf{Д1808})
	$$
		\dint x \ln{\dfrac{1 + x}{1-x}}dx
	$$
\end{problem}
\begin{proof}
	$$
		\dint x \ln{\dfrac{1 + x}{1-x}}dx = \dfrac{x^2}{2} \ln{\dfrac{1 + x}{1-x}} - \dint \dfrac{x^2}{2} \dfrac{2}{1 - x^2}dx \Rightarrow
	$$
	$$
		\Rightarrow -\dint \dfrac{x^2}{2} \dfrac{2}{1 - x^2}dx = \dint \dfrac{1 - 1 - x^2}{1-x^2}dx = -\dint\dfrac{1}{1-x^2}dx - x + C = -\dfrac{1}{2}\ln{\dfrac{1 + x}{1-x}}  - x + C \Rightarrow
	$$
	$$
		\Rightarrow	\dint x \ln{\dfrac{1 + x}{1-x}}dx = \dfrac{x^2}{2} \ln{\dfrac{1 + x}{1-x}} - \dfrac{1}{2}\ln{\dfrac{1 + x}{1-x}}  - x + C
	$$
\end{proof}

\begin{problem}(\textbf{Д1822})
	$$
		\dint e^{\sqrt{x}}dx
	$$
\end{problem}
\begin{proof}
	$$
		\dint e^{\sqrt{x}}dx = \left|u =\sqrt{x}, \, du = \dfrac{1}{2\sqrt{x}}dx\right| = \dint e^{u}2u du = 2 ue^u - \dint 2e^u du =
	$$
	$$	
		= 2 u e^u - 2 e^u + C = 2\sqrt{x}e^{\sqrt{x}} - 2e^{\sqrt{x}} + C
	$$
\end{proof}

\begin{problem}(\textbf{Д1837})
	$$
		\dint \dfrac{dx}{x^2 - x + 2}	
	$$
\end{problem}

\begin{proof}
	$$
		\dint \dfrac{dx}{x^2 - x + 2} = \dint \dfrac{dx}{\left(x - \frac{1}{2}\right)^2  + \frac{7}{4}} = \dint \dfrac{2d(2x - 1)}{\left(2x - 1\right)^2  + 7} = \dint \dfrac{2du}{u^2 + 7} = \dfrac{2}{\sqrt{7}}\arctg{\left(\dfrac{2x - 1}{\sqrt{7}}\right)} + C
	$$
\end{proof}

\begin{problem}(\textbf{Д1842})
	$$
		\dint \dfrac{x^3 dx}{x^4 - x^2 + 2} 
	$$
\end{problem}
\begin{proof}
	$$
		\dint \dfrac{x^3 dx}{x^4 - x^2 + 2} =|u = x^2, \, du = 2x dx = 2 \sqrt{u}dx| = \dfrac{1}{2}\dint \dfrac{u du}{u^2 - u + 2} = \dfrac{1}{2}\dint \dfrac{4u du}{(2u -1)^2 + 7} = 
	$$
	$$
		=	\dfrac{1}{2}\dint \dfrac{(2u - 1) + 1 }{(2u -1)^2 + 7}d(2u -1) = \dfrac{1}{2}\dint\dfrac{vdv}{v^2 + 7} + \dfrac{1}{2\sqrt{7}}\arctg{\left(\dfrac{2x^2 -1}{\sqrt{7}}\right)} =
	$$
	$$	
		= \dfrac{1}{4}\ln{|v^2 + 7|} + \dfrac{1}{2\sqrt{7}}\arctg{\left(\dfrac{2x^2 -1}{\sqrt{7}}\right)} + C = \dfrac{1}{4}\ln{|(2x^2 - 1)^2 + 7|} + \dfrac{1}{2\sqrt{7}}\arctg{\left(\dfrac{2x^2 -1}{\sqrt{7}}\right)} + C
	$$
\end{proof}

\begin{problem}(\textbf{Д1867})
	$$
		\dint \dfrac{xdx}{(x + 1)(x+2)(x+3)}
	$$
\end{problem}
\begin{proof}
	$$
		P(x) = x, \, Q(x) = (x + 1)(x+2)(x+3), \, Q'(x) = (x+ 2)(x+ 3) + (x+ 1)(x+ 3) + (x+1)(x+2)
	$$
	$$
		z_1 = -1,\, z_2 = -2, \, z_3 = -3, \, \deg{P} < \deg{Q} 
	$$
	$$
		\dfrac{P(z_1)}{Q'(z_1)} = \dfrac{-1}{2} = -\dfrac{1}{2}, \, \dfrac{P(z_2)}{Q'(z_2)} = \dfrac{-2}{-1} = 2, \, \dfrac{P(z_3)}{Q'(z_3)} = \dfrac{-3}{2} = - \dfrac{3}{2} 
	$$
	$$
		\dint \dfrac{xdx}{(x + 1)(x+2)(x+3)} = \dint -\dfrac{1}{2}{\cdot}\dfrac{1}{x+1} + 2\dfrac{1}{x + 2} - \dfrac{3}{2}{\cdot}\dfrac{1}{x + 3}dx = 
	$$
	$$
		= -\dfrac{1}{2}\ln{(x+1)} + 2\ln{(x + 2)} - \dfrac{3}{2}\ln{(x + 3)} + C = \dfrac{1}{2}\ln{\left(\dfrac{(x+2)^4}{(x+1)(x+3)^3}\right)} + C
	$$
\end{proof}
\begin{problem}(\textbf{Д1870})
	$$
		\dint \dfrac{x^4dx}{x^4 + 5x^2 + 4}
	$$
\end{problem}
\begin{proof}
	$$
		\dfrac{x^4}{x^4 + 5x^2 + 4} = 1 - \dfrac{5x^2 + 4}{(x^2 + 1)(x^2 + 4)} = 1 - \dfrac{A}{x^2 + 1} - \dfrac{B}{x^2 + 4} = 1 - \dfrac{Ax^2 + 4A + Bx^2 + B}{(x^2 + 1)(x^2 + 4)} \Rightarrow
	$$
	$$
		\Rightarrow 
		\left\{
		\begin{matrix}
			A &+& B & = & 5\\
			4A &+& B & = & 4
		\end{matrix}
		\right.
		\Rightarrow  A + 4 - 4A = 5 \Rightarrow 
		\left\{
			\begin{matrix}
				A &=& -\tfrac{1}{3}\\[5pt]
				B &=& \tfrac{16}{3}
			\end{matrix}
		\right.
	$$	
	$$
		\dint \dfrac{x^4dx}{x^4 + 5x^2 + 4} = x + \dfrac{1}{3}\dint \dfrac{dx}{x^2 + 1} - \dfrac{16}{3}\dint \dfrac{dx}{x^2 + 4} = x + \dfrac{1}{3}\arctg{x} - \dfrac{8}{3}\arctg{\left(\dfrac{x}{2}\right)} + C
	$$
\end{proof}

\begin{problem}(\textbf{Д1882})
	$$
		\dint \dfrac{xdx}{x^3 - 1}
	$$
\end{problem}
\begin{proof}
	$$
		\dfrac{x}{x^3 - 1} = \dfrac{x}{(x-1)(x^2+ x + 1)} = \dfrac{A}{x- 1} + \dfrac{Bx + C}{x^2 + x + 1} = \dfrac{Ax^2 + Ax + A + Bx^2 -Bx + Cx - C}{(x-1)(x^2+ x + 1)} \Rightarrow
	$$
	$$
		\Rightarrow 
		\left\{
			\begin{matrix}
				&&A &+& B &=& 0\\
				A &-& B &+& C &=& 1\\
				&&A &-& C &=& 0
			\end{matrix}
		\right.
		\Rightarrow 
		\left\{
			\begin{matrix}
				A &=& - B \\
				A &=& C\\
				3A &=& 1
			\end{matrix}
		\right.
		\Rightarrow 
		\left\{
			\begin{matrix}
				A &=& \tfrac{1}{3} \\[5pt]
				C &=& \tfrac{1}{3}\\[5pt]
				B &=& -\tfrac{1}{3}
			\end{matrix}
		\right.
	$$
	$$
		\dint \dfrac{xdx}{x^3 - 1} = \dfrac{1}{3}\dint \dfrac{dx}{x - 1}  -\dfrac{1}{3}\dint \dfrac{x - 1}{x^2 + x + 1}dx = \dfrac{1}{3}\ln{|x-1|} - \dfrac{1}{6}\dint \dfrac{2x + 1}{x^2 + x + 1}dx + \dfrac{1}{2}\dint \dfrac{dx}{x^2 + x + 1} =
	$$
	$$
		= \dfrac{1}{3}\ln{|x-1|} -\dfrac{1}{6}\ln{|x^2 + x + 1|} + \dfrac{1}{2}\dint\dfrac{dx}{\left(x + \frac{1}{2}\right)^2 + \frac{3}{4}} = \dfrac{1}{3}\ln{\left| \dfrac{x-1}{\sqrt{x^2 + x + 1}}\right|} + \dfrac{1}{\sqrt{3}}\arctg{\dfrac{2x + 1}{\sqrt{3}}} + C
	$$
\end{proof}
\end{document}