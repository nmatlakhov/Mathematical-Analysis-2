\documentclass[12pt]{article}
\usepackage[left=1cm, right=1cm, top=2cm,bottom=1.5cm]{geometry} 

\usepackage[parfill]{parskip}
\usepackage[utf8]{inputenc}
\usepackage[T2A]{fontenc}
\usepackage[russian]{babel}
\usepackage{enumitem}
\usepackage[normalem]{ulem}
\usepackage{amsfonts, amsmath, amsthm, amssymb, mathtools,xcolor}
\usepackage{blkarray}

\usepackage{tabularx}
\usepackage{hhline}

\usepackage{accents}
\usepackage{fancyhdr}
\pagestyle{fancy}
\renewcommand{\headrulewidth}{1.5pt}
\renewcommand{\footrulewidth}{1pt}

\usepackage{graphicx}
\usepackage[figurename=Рис.]{caption}
\usepackage{subcaption}
\usepackage{float}

%%Наименование папки откуда забирать изображения
\graphicspath{ {./images/} }

%%Изменение формата для ввода доказательства
\renewcommand{\proofname}{$\square$  \nopunct}
\renewcommand\qedsymbol{$\blacksquare$}

%%Изменение отступа на таблицах
\addto\captionsrussian{%
	\renewcommand{\proofname}{$\square$ \nopunct}%
}
%% Римские цифры
\newcommand{\RN}[1]{%
	\textup{\uppercase\expandafter{\romannumeral#1}}%
}

%% Для удобства записи
\newcommand{\MR}{\mathbb{R}}
\newcommand{\MC}{\mathbb{C}}
\newcommand{\MQ}{\mathbb{Q}}
\newcommand{\MN}{\mathbb{N}}
\newcommand{\MZ}{\mathbb{Z}}
\newcommand{\MTB}{\mathbb{T}}
\newcommand{\MTI}{\mathbb{I}}
\newcommand{\MI}{\mathrm{I}}
\newcommand{\MCI}{\mathcal{I}}
\newcommand{\MJ}{\mathrm{J}}
\newcommand{\MH}{\mathrm{H}}
\newcommand{\MT}{\mathrm{T}}
\newcommand{\MU}{\mathcal{U}}
\newcommand{\MV}{\mathcal{V}}
\newcommand{\MB}{\mathcal{B}}
\newcommand{\MF}{\mathcal{F}}
\newcommand{\MW}{\mathcal{W}}
\newcommand{\ML}{\mathcal{L}}
\newcommand{\MP}{\mathcal{P}}
\newcommand{\VN}{\varnothing}
\newcommand{\VE}{\varepsilon}
\newcommand{\dx}{\, dx}
\newcommand{\dy}{\, dy}
\newcommand{\dz}{\, dz}
\newcommand{\dd}{\, d}


\theoremstyle{definition}
\newtheorem{defn}{Опр:}
\newtheorem{rem}{Rm:}
\newtheorem{prop}{Утв.}
\newtheorem{exrc}{Упр.}
\newtheorem{problem}{Задача}
\newtheorem{lemma}{Лемма}
\newtheorem{theorem}{Теорема}
\newtheorem{corollary}{Следствие}

\newenvironment{cusdefn}[1]
{\renewcommand\thedefn{#1}\defn}
{\enddefn}

\DeclareRobustCommand{\divby}{%
	\mathrel{\text{\vbox{\baselineskip.65ex\lineskiplimit0pt\hbox{.}\hbox{.}\hbox{.}}}}%
}
\DeclareRobustCommand{\ndivby}{\mkern-1mu\not\mathrel{\mkern4.5mu\divby}\mkern1mu}


%Короткий минус
\DeclareMathSymbol{\SMN}{\mathbin}{AMSa}{"39}
%Длинная шапка
\newcommand{\overbar}[1]{\mkern 1.5mu\overline{\mkern-1.5mu#1\mkern-1.5mu}\mkern 1.5mu}
%Функция знака
\DeclareMathOperator{\sgn}{sgn}

%Функция ранга
\DeclareMathOperator{\rk}{\text{rk}}
\DeclareMathOperator{\diam}{\text{diam}}


%Обозначение константы
\DeclareMathOperator{\const}{\text{const}}

\DeclareMathOperator{\codim}{\text{codim}}

\DeclareMathOperator*{\dsum}{\displaystyle\sum}
\newcommand{\ddsum}[2]{\displaystyle\sum\limits_{#1}^{#2}}

%Интеграл в большом формате
\DeclareMathOperator{\dint}{\displaystyle\int}
\newcommand{\ddint}[2]{\displaystyle\int\limits_{#1}^{#2}}
\newcommand{\ssum}[1]{\displaystyle \sum\limits_{n=1}^{\infty}{#1}_n}

\newcommand{\smallerrel}[1]{\mathrel{\mathpalette\smallerrelaux{#1}}}
\newcommand{\smallerrelaux}[2]{\raisebox{.1ex}{\scalebox{.75}{$#1#2$}}}

\newcommand{\smallin}{\smallerrel{\in}}
\newcommand{\smallnotin}{\smallerrel{\notin}}

\newcommand*{\medcap}{\mathbin{\scalebox{1.25}{\ensuremath{\cap}}}}%
\newcommand*{\medcup}{\mathbin{\scalebox{1.25}{\ensuremath{\cup}}}}%

\makeatletter
\newcommand{\vast}{\bBigg@{3.5}}
\newcommand{\Vast}{\bBigg@{5}}
\makeatother

%Промежуточное значение для sup\inf, поскольку они имеют разную высоту
\newcommand{\newsup}{\mathop{\smash{\mathrm{sup}}}}
\newcommand{\newinf}{\mathop{\mathrm{inf}\vphantom{\mathrm{sup}}}}

%Скалярное произведение
\newcommand{\inner}[2]{\left\langle #1, #2 \right\rangle }
\newcommand{\linsp}[1]{\left\langle #1 \right\rangle }
\newcommand{\linmer}[2]{\left\langle #1 \vert #2\right\rangle }

%Подпись символов снизу
\newcommand{\ubar}[1]{\underaccent{\bar}{#1}}

%% Шапка для букв сверху
\newcommand{\wte}[1]{\widetilde{#1}}
\newcommand{\wht}[1]{\widehat{#1}}
\newcommand{\ovl}[1]{\overline{#1}}

%%Трансформация Фурье
\newcommand{\fourt}[1]{\mathcal{F}\left(#1\right)}
\newcommand{\ifourt}[1]{\mathcal{F}^{-1}\left(#1\right)}

%%Символ вектора
\newcommand{\vecm}[1]{\overrightarrow{#1\,}}

%%Пространстов матриц
\newcommand{\matsq}[1]{\operatorname{Mat}_{#1}}
\newcommand{\mat}[2]{\operatorname{Mat}_{#1, #2}}

%Оператор для действ и мнимых чисел
\DeclareMathOperator{\IM}{\operatorname{Im}}
\DeclareMathOperator{\RE}{\operatorname{Re}}
\DeclareMathOperator{\li}{\operatorname{li}}
\DeclareMathOperator{\GL}{\operatorname{GL}}
\DeclareMathOperator{\SL}{\operatorname{SL}}
\DeclareMathOperator{\Char}{\operatorname{char}}
\DeclareMathOperator\Arg{Arg}

%Делимость чисел
\newcommand{\modn}[3]{#1 \equiv #2 \; (\bmod \; #3)}


%%Взятие в скобки, модули и норму
\newcommand{\parfit}[1]{\left( #1 \right)}
\newcommand{\modfit}[1]{\left| #1 \right|}
\newcommand{\sqparfit}[1]{\left\{ #1 \right\}}
\newcommand{\normfit}[1]{\left\| #1 \right\|}

%%Функция для обозначения равномерной сходимости по множеству
\newcommand{\uconv}[1]{\overset{#1}{\rightrightarrows}}
\newcommand{\uconvm}[2]{\overset{#1}{\underset{#2}{\rightrightarrows}}}


%%Функция для обозначения нижнего и верхнего интегралов
\def\upint{\mathchoice%
	{\mkern13mu\overline{\vphantom{\intop}\mkern7mu}\mkern-20mu}%
	{\mkern7mu\overline{\vphantom{\intop}\mkern7mu}\mkern-14mu}%
	{\mkern7mu\overline{\vphantom{\intop}\mkern7mu}\mkern-14mu}%
	{\mkern7mu\overline{\vphantom{\intop}\mkern7mu}\mkern-14mu}%
	\int}
\def\lowint{\mkern3mu\underline{\vphantom{\intop}\mkern7mu}\mkern-10mu\int}

%%След матрицы
\DeclareMathOperator*{\tr}{tr}

\makeatletter
\renewcommand*\env@matrix[1][*\c@MaxMatrixCols c]{%
	\hskip -\arraycolsep
	\let\@ifnextchar\new@ifnextchar
	\array{#1}}
\makeatother


%% Переопределение функции хи, чтобы выглядела более приятно
\makeatletter
\@ifdefinable\@latex@chi{\let\@latex@chi\chi}
\renewcommand*\chi{{\@latex@chi\smash[t]{\mathstrut}}} % want only bottom half of \mathstrut
\makeatletter

\begin{document}
\lhead{Математический анализ - \RN{2}}
\chead{Косухин О.Н.}
\rhead{Семинар - 8: ДЗ}

\section*{Определенный интеграл}

\textbf{ДЗ}: $2182$ б), $2189$ с подсказкой, $2192^*$ - необязательная и трудная, $2193.2$ (картинкой решается легко), $2200$, $2205$, $2207$, $2209$, $2216$ в).

\begin{problem}(\textbf{Д2182} б))
	Найти нижнюю и верхнюю интегральные суммы на соответствующих сегментах, деля их на $n$ равных частей:
	$$
		f(x) = \sqrt{x}, \, 0 \leq x \leq 1
	$$
\end{problem}

\begin{proof}
	Делим отрезок на $n$ равных частей длины $\tfrac{1 - 0}{n} = \tfrac{1}{n}$, выберем отмеченные точки слева, тогда:
	$$
		x_0 = 0, x_1 = 0 + \dfrac{1}{n}, \dotsc, x_j = 0 + \dfrac{j}{n}, \dotsc, x_n = 1 \Rightarrow \lambda(\MTB) = \dfrac{1}{n}
	$$
	Посчитаем суммы Дарбу:
	$$
		S(f,\MTB) = f(x_1)\dfrac{1}{n} + f(x_2)\dfrac{1}{n} + \dotsc + f(x_n)\dfrac{1}{n}		
	$$
	$$
		s(f,\MTB) = f(x_0)\dfrac{1}{n} + f(x_1)\dfrac{1}{n} + \dotsc + f(x_{n-1})\dfrac{1}{n}		
	$$
	$$
		\Omega(\MTB) = S(f,\MTB) - s(f,\MTB) = (f(x_n) - f(x_0)){\cdot}\dfrac{1}{n} = (f(1) - f(0))\dfrac{1}{n} = \dfrac{1}{n} \to 0
	$$
	$$
		s(f,\MTB) = \dfrac{1}{n}{\cdot}\left(\sqrt{0} + \sqrt{\dfrac{1}{n}} + \sqrt{\dfrac{2}{n}} + \dotsc + \sqrt{\dfrac{n-1}{n}}\right)
	$$
	$$
		S(f,\MTB) = \dfrac{1}{n}{\cdot}\left(\sqrt{\dfrac{1}{n}} + \sqrt{\dfrac{2}{n}} + \dotsc + \sqrt{\dfrac{n-1}{n}} + 1\right)
	$$
\end{proof}

\begin{problem}(\textbf{Д2189})
	Вычислить определенные интегралы, рассматривая их как пределы соответствующих интегральных сумм и производя разбиение промежутка интеграции надлежащим образом:
	$$
		\ddint{a}{b}\dfrac{dx}{x^2}, \, 0 < a < b
	$$
	Положить $\xi_i = \sqrt{x_i x_{i+1}}, \, i = 0,1, \dotsc, n-1$.
\end{problem}
\begin{proof}
	Делим отрезок на $n$ равных частей длины $\tfrac{b - a}{n}$, тогда:
	$$
		a = x_0 < x_1 < \dotsc < x_n = b
	$$
	$$
		\xi_i = \sqrt{x_i x_{i+1}}, \, i = 0, 1 , \dotsc, n, \, \xi_i \in (x_i, x_{i+1})
	$$
	Посчитаем интегральную сумму:
	$$
		\sigma(f,\MTB, \xi) = \ddsum{i = 0}{n-1}f(\xi_i){\cdot}(x_{i + 1} - x_{i}) = \ddsum{i = 0}{n-1}\dfrac{1}{x_i x_{i+1}}(x_{i+1} - x_i) = \ddsum{i = 0}{n-1}\left(\dfrac{1}{x_i} - \dfrac{1}{x_{i+1}}\right) = \dfrac{1}{x_0} - \dfrac{1}{x_{n}} = \dfrac{b - a}{ab}
	$$
\end{proof}
\newpage

\begin{problem}(\textbf{Д2192})
	Вычислить интеграл Пуассона:
	$$
		\ddint{0}{\pi}\ln(1 - 2\alpha \cos(x) + \alpha^2)dx
	$$
	При а) $|\alpha| <1$; б) $|\alpha| > 1$;
\end{problem}
\begin{proof}
	Делим отрезок на $n$ равных частей длины $\tfrac{\pi - 0}{n} = \tfrac{\pi}{n}$, тогда:
	$$
		0 = x_0, x_1 = 0 + \dfrac{\pi}{n}, x_2 = \dfrac{\pi}{n} + \dfrac{\pi}{n}, \dotsc, x_j = j{\cdot}\dfrac{\pi}{n}, \dotsc, x_n = \pi \Rightarrow \lambda(\MTB) = \dfrac{\pi}{n}
	$$
	Используем формулу Эйлера:
	$$
		z = e^{ix} = \cos{x} + i\sin{x}, \, \ovl{z} = \cos{x} - i \sin{x} \Rightarrow 
	$$
	$$	
		\Rightarrow (\alpha - z)(\alpha - \ovl{z}) = \alpha^2 - \alpha(z + \ovl{z}) + z\ovl{z} = \alpha^2 - 2\alpha\cos{x} + 1
	$$
	Выберем отмеченные точки слева, тогда интегральная сумма будет равна:
	$$
		\sigma(f,\MTB,\xi) = \ddsum{j = 0}{n-1}\ln\left(\alpha - e^{\tfrac{i\pi }{n}j} \right){\cdot}\left(\alpha - e^{\tfrac{-i\pi }{n}j} \right){\cdot}\dfrac{\pi}{n} = \dfrac{\pi}{n}{\cdot}\ln\prod\limits_{j = 0}^{n-1}\left(\alpha - e^{\tfrac{i\pi }{n}j} \right){\cdot}\left(\alpha - e^{\tfrac{-i\pi }{n}j} \right)
	$$
	Заметим, что в произведении находятся единичные корни уравнения $\alpha^{2n} - 1 =0$, два корня $1$ и отсутствует корень $-1$, тогда:
	$$
		\sigma(f,\MTB,\xi) =\dfrac{\pi}{n}{\cdot}\ln\prod\limits_{j = 0}^{n-1}\left(\alpha - e^{\tfrac{i\pi }{n}j} \right){\cdot}\left(\alpha - e^{\tfrac{-i\pi }{n}j} \right) = \dfrac{\pi}{n}{\cdot}\ln\dfrac{(\alpha - 1)(\alpha^{2n} - 1)}{\alpha + 1}  
	$$
	Тогда, если $|\alpha|<1$, то $\alpha^{2n} \to 0$ при $n\to \infty$:
	$$
		\dfrac{(\alpha - 1)(\alpha^{2n} - 1)}{\alpha + 1}  \xrightarrow[n \to \infty]{} \dfrac{1 - \alpha}{1 + \alpha} > 0 \Rightarrow \sigma(f,\MTB,\xi) \to 0
	$$
	Если $|\alpha| > 1$, то тогда:
	$$
		\dfrac{\pi}{n}{\cdot}\ln\dfrac{(\alpha - 1)(\alpha^{2n} - 1)}{\alpha + 1}  = \dfrac{\pi}{n}{\cdot}\left(2n\ln(\alpha) + \ln\dfrac{(\alpha -1)(1 - \alpha^{-2n})}{\alpha + 1}\right) \xrightarrow[n \to \infty]{} 2\pi\ln(\alpha)
	$$
\end{proof}

\begin{problem}(\textbf{Д2193.2})
	Пусть функция $f(x)$ выпукла сверху на сегменте $[a,b]$. Доказать, что:
	$$
		(b-a)\dfrac{f(a) + f(b)}{2} \leq \ddint{a}{b}f(x)dx \leq (b-a)f\left(\dfrac{a+b}{2}\right)
	$$
\end{problem}
\begin{proof}
	Функция $f(x)$ выпукла сверху на $[a,b]$ означает:
	$$
		\forall \lambda \in [0,1], \, \forall x_1,x_2 \in [a,b], \, f(\lambda x_1 + (1-\lambda)x_2) \geq \lambda f(x_1) + (1-\lambda)f(x_2)
	$$
	Покажем, что $f(x)$ - непрерывна на $(a,b)$. Пусть $x_0 \in (a,b)$ и $0 < h < h_0$ и $x_1 = x_0, \, x_2 = x_0 + h_0$, тогда:
	$$
		x = x_0 + h = \lambda x_1 + (1-\lambda)x_2 \Rightarrow \lambda = \dfrac{x_2 - x}{x_2 - x_1} = \dfrac{h_0 - h}{h_0} \in (0,1) \Rightarrow
	$$
	$$
		\Rightarrow f(x_0 + h) \geq \lambda f(x_1) + (1-\lambda)f(x_2) = \dfrac{h_0 - h}{h_0}f(x_0) + \dfrac{h}{h_0}f(x_0 + h_0) \Rightarrow 
	$$
	$$
		\Rightarrow f(x_0 + h) - f(x_0) \geq h\dfrac{f(x_0 + h_0) - f(x_0)}{h_0}
	$$
	Функция выпукла сверху на сегменте $[a,b]$, тогда верно:
	$$
		f\left(\dfrac{a + b}{2}\right) \geq \dfrac{f(a) + f(b)}{2}
	$$
\end{proof}


\begin{problem}(\textbf{Д2209})
	$$
		\ddint{-1/2}{1/2}\dfrac{dx}{\sqrt{1-x^2}}
	$$
\end{problem}
\begin{proof}
	Функция непрерывна на всем интервале $\left[-\tfrac{1}{2},\tfrac{1}{2}\right]$, воспользуемся формулой Ньютона-Лейбница:
	$$
		\ddint{-1/2}{1/2}\dfrac{dx}{\sqrt{1-x^2}} = \arcsin\left(\tfrac{1}{2}\right) - \arcsin\left(-\tfrac{1}{2}\right) =\dfrac{\pi}{6} + \dfrac{\pi}{6} = \dfrac{\pi}{3}
	$$
\end{proof}

\begin{problem}(\textbf{Д2216} в))
	Объяснить, почему формальное применение формулы Ньютона-Лейбница приводит к неверным результатам, если:
	$$
		\ddint{-1}{1}\dfrac{d}{dx}\left(\arctg\dfrac{1}{x}\right)dx
	$$
\end{problem}
\begin{proof}
	Функция $F(x) = \arctg\dfrac{1}{x}$
\end{proof}

\end{document}